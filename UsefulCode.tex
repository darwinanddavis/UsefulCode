\documentclass[10,portrait]{article}
\usepackage{lmodern}
\usepackage{amssymb,amsmath}
\usepackage{ifxetex,ifluatex}
\usepackage{fixltx2e} % provides \textsubscript
\ifnum 0\ifxetex 1\fi\ifluatex 1\fi=0 % if pdftex
  \usepackage[T1]{fontenc}
  \usepackage[utf8]{inputenc}
\else % if luatex or xelatex
  \ifxetex
    \usepackage{mathspec}
  \else
    \usepackage{fontspec}
  \fi
  \defaultfontfeatures{Ligatures=TeX,Scale=MatchLowercase}
\fi
% use upquote if available, for straight quotes in verbatim environments
\IfFileExists{upquote.sty}{\usepackage{upquote}}{}
% use microtype if available
\IfFileExists{microtype.sty}{%
\usepackage[]{microtype}
\UseMicrotypeSet[protrusion]{basicmath} % disable protrusion for tt fonts
}{}
\PassOptionsToPackage{hyphens}{url} % url is loaded by hyperref
\usepackage[unicode=true]{hyperref}
\PassOptionsToPackage{usenames,dvipsnames}{color} % color is loaded by hyperref
\hypersetup{
            pdftitle={Useful R code},
            colorlinks=true,
            linkcolor=pink,
            citecolor=red,
            urlcolor=blue,
            breaklinks=true}
\urlstyle{same}  % don't use monospace font for urls
\usepackage[margin=1in]{geometry}
\usepackage[]{biblatex}
\usepackage{color}
\usepackage{fancyvrb}
\newcommand{\VerbBar}{|}
\newcommand{\VERB}{\Verb[commandchars=\\\{\}]}
\DefineVerbatimEnvironment{Highlighting}{Verbatim}{commandchars=\\\{\}}
% Add ',fontsize=\small' for more characters per line
\usepackage{framed}
\definecolor{shadecolor}{RGB}{248,248,248}
\newenvironment{Shaded}{\begin{snugshade}}{\end{snugshade}}
\newcommand{\KeywordTok}[1]{\textcolor[rgb]{0.13,0.29,0.53}{\textbf{#1}}}
\newcommand{\DataTypeTok}[1]{\textcolor[rgb]{0.13,0.29,0.53}{#1}}
\newcommand{\DecValTok}[1]{\textcolor[rgb]{0.00,0.00,0.81}{#1}}
\newcommand{\BaseNTok}[1]{\textcolor[rgb]{0.00,0.00,0.81}{#1}}
\newcommand{\FloatTok}[1]{\textcolor[rgb]{0.00,0.00,0.81}{#1}}
\newcommand{\ConstantTok}[1]{\textcolor[rgb]{0.00,0.00,0.00}{#1}}
\newcommand{\CharTok}[1]{\textcolor[rgb]{0.31,0.60,0.02}{#1}}
\newcommand{\SpecialCharTok}[1]{\textcolor[rgb]{0.00,0.00,0.00}{#1}}
\newcommand{\StringTok}[1]{\textcolor[rgb]{0.31,0.60,0.02}{#1}}
\newcommand{\VerbatimStringTok}[1]{\textcolor[rgb]{0.31,0.60,0.02}{#1}}
\newcommand{\SpecialStringTok}[1]{\textcolor[rgb]{0.31,0.60,0.02}{#1}}
\newcommand{\ImportTok}[1]{#1}
\newcommand{\CommentTok}[1]{\textcolor[rgb]{0.56,0.35,0.01}{\textit{#1}}}
\newcommand{\DocumentationTok}[1]{\textcolor[rgb]{0.56,0.35,0.01}{\textbf{\textit{#1}}}}
\newcommand{\AnnotationTok}[1]{\textcolor[rgb]{0.56,0.35,0.01}{\textbf{\textit{#1}}}}
\newcommand{\CommentVarTok}[1]{\textcolor[rgb]{0.56,0.35,0.01}{\textbf{\textit{#1}}}}
\newcommand{\OtherTok}[1]{\textcolor[rgb]{0.56,0.35,0.01}{#1}}
\newcommand{\FunctionTok}[1]{\textcolor[rgb]{0.00,0.00,0.00}{#1}}
\newcommand{\VariableTok}[1]{\textcolor[rgb]{0.00,0.00,0.00}{#1}}
\newcommand{\ControlFlowTok}[1]{\textcolor[rgb]{0.13,0.29,0.53}{\textbf{#1}}}
\newcommand{\OperatorTok}[1]{\textcolor[rgb]{0.81,0.36,0.00}{\textbf{#1}}}
\newcommand{\BuiltInTok}[1]{#1}
\newcommand{\ExtensionTok}[1]{#1}
\newcommand{\PreprocessorTok}[1]{\textcolor[rgb]{0.56,0.35,0.01}{\textit{#1}}}
\newcommand{\AttributeTok}[1]{\textcolor[rgb]{0.77,0.63,0.00}{#1}}
\newcommand{\RegionMarkerTok}[1]{#1}
\newcommand{\InformationTok}[1]{\textcolor[rgb]{0.56,0.35,0.01}{\textbf{\textit{#1}}}}
\newcommand{\WarningTok}[1]{\textcolor[rgb]{0.56,0.35,0.01}{\textbf{\textit{#1}}}}
\newcommand{\AlertTok}[1]{\textcolor[rgb]{0.94,0.16,0.16}{#1}}
\newcommand{\ErrorTok}[1]{\textcolor[rgb]{0.64,0.00,0.00}{\textbf{#1}}}
\newcommand{\NormalTok}[1]{#1}
\IfFileExists{parskip.sty}{%
\usepackage{parskip}
}{% else
\setlength{\parindent}{0pt}
\setlength{\parskip}{6pt plus 2pt minus 1pt}
}
\setlength{\emergencystretch}{3em}  % prevent overfull lines
\providecommand{\tightlist}{%
  \setlength{\itemsep}{0pt}\setlength{\parskip}{0pt}}
\setcounter{secnumdepth}{0}
% Redefines (sub)paragraphs to behave more like sections
\ifx\paragraph\undefined\else
\let\oldparagraph\paragraph
\renewcommand{\paragraph}[1]{\oldparagraph{#1}\mbox{}}
\fi
\ifx\subparagraph\undefined\else
\let\oldsubparagraph\subparagraph
\renewcommand{\subparagraph}[1]{\oldsubparagraph{#1}\mbox{}}
\fi

% set default figure placement to htbp
\makeatletter
\def\fps@figure{htbp}
\makeatother


\title{Useful R code}
\author{Matthew Malishev\textsuperscript{1}*\\
\emph{\textsuperscript{1} Department of Biology, Emory University, 1510
Clifton Road NE, Atlanta, GA, USA, 30322}}
\date{}

\begin{document}
\maketitle

{
\hypersetup{linkcolor=black}
\setcounter{tocdepth}{3}
\tableofcontents
}
~

Date: 2018-08-16\\
R version: 3.5.0\\
*Corresponding author:
\href{mailto:matthew.malishev@gmail.com}{\nolinkurl{matthew.malishev@gmail.com}}\\
This document can be found at
\url{https://github.com/darwinanddavis/UsefulCode}

\newpage  

\subsection{Overview}\label{overview}

This document outlines some useful R code for plotting, cool functions,
and other random tidbits.

\subsubsection{Install dependencies}\label{install-dependencies}

\begin{Shaded}
\begin{Highlighting}[]
\NormalTok{packages <-}\StringTok{ }\KeywordTok{c}\NormalTok{(}\StringTok{"rgdal"}\NormalTok{,}\StringTok{"dplyr"}\NormalTok{,}\StringTok{"zoo"}\NormalTok{,}\StringTok{"RColorBrewer"}\NormalTok{,}\StringTok{"viridis"}\NormalTok{,}\StringTok{"plyr"}\NormalTok{,}\StringTok{"digitize"}\NormalTok{,}\StringTok{"jpeg"}\NormalTok{,}\StringTok{"devtools"}\NormalTok{,}\StringTok{"imager"}\NormalTok{,}\StringTok{"dplyr"}\NormalTok{,}\StringTok{"ggplot2"}\NormalTok{,}\StringTok{"ggridges"}\NormalTok{,}\StringTok{"ggjoy"}\NormalTok{,}\StringTok{"ggthemes"}\NormalTok{,}\StringTok{"svDialogs"}\NormalTok{,}\StringTok{"data.table"}\NormalTok{,}\StringTok{"tibble"}\NormalTok{,}\StringTok{"extrafont"}\NormalTok{,}\StringTok{"sp"}\NormalTok{)   }
\ControlFlowTok{if}\NormalTok{ (}\KeywordTok{require}\NormalTok{(packages)) \{}
    \KeywordTok{install.packages}\NormalTok{(packages,}\DataTypeTok{dependencies =}\NormalTok{ T)}
    \KeywordTok{require}\NormalTok{(packages)}
\NormalTok{\}}
\KeywordTok{lapply}\NormalTok{(packages,library,}\DataTypeTok{character.only=}\NormalTok{T)}
\end{Highlighting}
\end{Shaded}

\subsubsection{Classes}\label{classes}

Convert character to factor to numeric without conversion error

\begin{Shaded}
\begin{Highlighting}[]
\KeywordTok{read.table}\NormalTok{(f,}\DataTypeTok{header=}\NormalTok{T,}\DataTypeTok{sep=}\StringTok{","}\NormalTok{,}\DataTypeTok{row.names=}\OtherTok{NULL}\NormalTok{,}\DataTypeTok{stringsAsFactors=}\OtherTok{FALSE}\NormalTok{, }\DataTypeTok{strip.white=}\OtherTok{TRUE}\NormalTok{)}
\NormalTok{f}\OperatorTok{$}\NormalTok{V2<-}\KeywordTok{as.numeric}\NormalTok{(f}\OperatorTok{$}\NormalTok{V2)}
\end{Highlighting}
\end{Shaded}

See call options for class

\begin{Shaded}
\begin{Highlighting}[]
\KeywordTok{methods}\NormalTok{(}\DataTypeTok{class=}\StringTok{"estUDm"}\NormalTok{)}
\end{Highlighting}
\end{Shaded}

Set dynamic input for variable / assign variable to char vector

\begin{Shaded}
\begin{Highlighting}[]
\NormalTok{shadedens<-}\ControlFlowTok{function}\NormalTok{(shadedens)\{ }\CommentTok{# set shade density to clumped (to match food) or sparse }
  \ControlFlowTok{if}\NormalTok{ (shadedens }\OperatorTok{==}\StringTok{ "Random"}\NormalTok{)\{}
    \KeywordTok{NLCommand}\NormalTok{(}\StringTok{"set Shade-density }\CharTok{\textbackslash{}"}\StringTok{Random}\CharTok{\textbackslash{}"}\StringTok{ "}\NormalTok{) }
\NormalTok{    \}}\ControlFlowTok{else}\NormalTok{\{}
    \KeywordTok{NLCommand}\NormalTok{(}\StringTok{"set Shade-density }\CharTok{\textbackslash{}"}\StringTok{Clumped}\CharTok{\textbackslash{}"}\StringTok{ "}\NormalTok{) }
\NormalTok{    \}}
\NormalTok{  \}}
\KeywordTok{shadedens}\NormalTok{(}\StringTok{"Clumped"}\NormalTok{) }\CommentTok{# set clumped resources}
\end{Highlighting}
\end{Shaded}

\subsubsection{Dataframes}\label{dataframes}

Optimal empty data frame

\begin{Shaded}
\begin{Highlighting}[]
\NormalTok{df <-}\StringTok{ }\KeywordTok{data.frame}\NormalTok{(}\DataTypeTok{Date=}\KeywordTok{as.Date}\NormalTok{(}\KeywordTok{character}\NormalTok{()),}
                 \DataTypeTok{X=}\KeywordTok{numeric}\NormalTok{(), }
                 \DataTypeTok{Y=}\KeywordTok{integer}\NormalTok{(), }
                 \DataTypeTok{stringsAsFactors=}\OtherTok{FALSE}\NormalTok{) }
\end{Highlighting}
\end{Shaded}

Add df cols with \texttt{mutate}

\begin{Shaded}
\begin{Highlighting}[]
\NormalTok{df <-}\StringTok{ }\KeywordTok{data.frame}\NormalTok{(}\StringTok{"a"}\NormalTok{=}\KeywordTok{rnorm}\NormalTok{(}\DecValTok{10}\NormalTok{),}\StringTok{"b"}\NormalTok{=(}\DecValTok{1}\OperatorTok{:}\DecValTok{20}\NormalTok{))}
\NormalTok{df }\OperatorTok\StringTok{ }
\StringTok{  }\KeywordTok{mutate}\NormalTok{(}
  \StringTok{"c"}\NormalTok{=}\KeywordTok{rnorm}\NormalTok{(}\DecValTok{20}\NormalTok{),}
  \DataTypeTok{b =}\NormalTok{ b }\OperatorTok{*}\DecValTok{67}
\NormalTok{  )}
\end{Highlighting}
\end{Shaded}

\subsubsection{\texorpdfstring{\texttt{ggplot}
functions}{ggplot functions}}\label{ggplot-functions}

Remove annoying stock gridlines from plot window

\begin{Shaded}
\begin{Highlighting}[]
\NormalTok{plot }\OperatorTok{+}\StringTok{ }\KeywordTok{theme_bw}\NormalTok{() }\OperatorTok{+}\StringTok{ }
\StringTok{  }\KeywordTok{theme}\NormalTok{(}\DataTypeTok{panel.border =} \KeywordTok{element_blank}\NormalTok{(), }\DataTypeTok{panel.grid.major =} \KeywordTok{element_blank}\NormalTok{(),}
                            \DataTypeTok{panel.grid.minor =} \KeywordTok{element_blank}\NormalTok{(), }\DataTypeTok{axis.line =} \KeywordTok{element_line}\NormalTok{(}\DataTypeTok{colour =} \StringTok{"black"}\NormalTok{))}
\CommentTok{# alternative (after loading ggridges library)}
\KeywordTok{theme_ridges}\NormalTok{(}\DataTypeTok{grid=}\NormalTok{F,}\DataTypeTok{center_axis_labels =}\NormalTok{ T)}
\end{Highlighting}
\end{Shaded}

Setting global graphics theme for ggplot

\begin{Shaded}
\begin{Highlighting}[]
\NormalTok{plot_it_gg <-}\StringTok{ }\ControlFlowTok{function}\NormalTok{(bg,family)\{ }\CommentTok{# bg = colour to plot bg, family = font family}
  \KeywordTok{theme_tufte}\NormalTok{(}\DataTypeTok{base_family =}\NormalTok{ family) }\OperatorTok{+}
\StringTok{  }\KeywordTok{theme}\NormalTok{(}\DataTypeTok{panel.border =} \KeywordTok{element_blank}\NormalTok{(),}
        \DataTypeTok{panel.grid.major =} \KeywordTok{element_blank}\NormalTok{(),}
        \DataTypeTok{panel.grid.minor =} \KeywordTok{element_blank}\NormalTok{(),}
        \DataTypeTok{panel.background =} \KeywordTok{element_rect}\NormalTok{(}\DataTypeTok{fill =}\NormalTok{ bg,}
                                        \DataTypeTok{colour =}\NormalTok{ bg),}
        \DataTypeTok{plot.background =} \KeywordTok{element_rect}\NormalTok{(}\DataTypeTok{fill=}\NormalTok{bg)}
\NormalTok{  ) }\OperatorTok{+}
\StringTok{    }\KeywordTok{theme}\NormalTok{(}\DataTypeTok{axis.line =} \KeywordTok{element_line}\NormalTok{(}\DataTypeTok{color =} \StringTok{"white"}\NormalTok{)) }\OperatorTok{+}
\StringTok{    }\KeywordTok{theme}\NormalTok{(}\DataTypeTok{axis.ticks =} \KeywordTok{element_line}\NormalTok{(}\DataTypeTok{color =} \StringTok{"white"}\NormalTok{)) }\OperatorTok{+}
\StringTok{    }\KeywordTok{theme}\NormalTok{(}\DataTypeTok{plot.title =} \KeywordTok{element_text}\NormalTok{(}\DataTypeTok{colour =} \StringTok{"white"}\NormalTok{)) }\OperatorTok{+}
\StringTok{    }\KeywordTok{theme}\NormalTok{(}\DataTypeTok{axis.title.x =} \KeywordTok{element_text}\NormalTok{(}\DataTypeTok{colour =} \StringTok{"white"}\NormalTok{), }
          \DataTypeTok{axis.title.y =} \KeywordTok{element_text}\NormalTok{(}\DataTypeTok{colour =} \StringTok{"white"}\NormalTok{)) }\OperatorTok{+}
\StringTok{    }\KeywordTok{theme}\NormalTok{(}\DataTypeTok{axis.text.x =} \KeywordTok{element_text}\NormalTok{(}\DataTypeTok{color =} \StringTok{"white"}\NormalTok{),}
          \DataTypeTok{axis.text.y =} \KeywordTok{element_text}\NormalTok{(}\DataTypeTok{color =} \StringTok{"white"}\NormalTok{)) }\OperatorTok{+}
\StringTok{    }\KeywordTok{theme}\NormalTok{(}\DataTypeTok{legend.key =} \KeywordTok{element_rect}\NormalTok{(}\DataTypeTok{fill =}\NormalTok{ bg)) }\OperatorTok{+}\StringTok{ }\CommentTok{# fill bg of legend}
\StringTok{    }\KeywordTok{theme}\NormalTok{(}\DataTypeTok{legend.title =} \KeywordTok{element_text}\NormalTok{(}\DataTypeTok{colour=}\StringTok{"white"}\NormalTok{)) }\OperatorTok{+}\StringTok{ }\CommentTok{# legend title}
\StringTok{    }\KeywordTok{theme}\NormalTok{(}\DataTypeTok{legend.text =} \KeywordTok{element_text}\NormalTok{(}\DataTypeTok{colour=}\StringTok{"white"}\NormalTok{)) }\CommentTok{# legend labels}
\NormalTok{\} }
\end{Highlighting}
\end{Shaded}

Put plot in function to take dynamic data inputs\\
Ref:
\url{http://jcborras.net/carpet/visualizing-political-divergences-2012-local-elections-in-helsinki.html}

\begin{Shaded}
\begin{Highlighting}[]
\NormalTok{hr.mass.plot <-}\StringTok{ }\ControlFlowTok{function}\NormalTok{(d) \{}
\NormalTok{  p <-}\StringTok{ }\KeywordTok{ggplot}\NormalTok{(d, }\KeywordTok{aes}\NormalTok{(HR, Mass, }\DataTypeTok{color =}\NormalTok{ colfunc)) }\OperatorTok{+}\StringTok{ }
\StringTok{    }\KeywordTok{geom_density_2d}\NormalTok{(}\DataTypeTok{data=}\NormalTok{d, }\KeywordTok{aes}\NormalTok{(}\DataTypeTok{x =}\NormalTok{ HR, }\DataTypeTok{y =}\NormalTok{ Mass), }
                    \DataTypeTok{stat =} \StringTok{"density2d"}\NormalTok{,}\DataTypeTok{position=}\StringTok{"identity"}\NormalTok{, }
                    \DataTypeTok{color=}\KeywordTok{adjustcolor}\NormalTok{(}\StringTok{"orange"}\NormalTok{,}\DataTypeTok{alpha=}\FloatTok{0.8}\NormalTok{), }\DataTypeTok{size=}\FloatTok{1.5}\NormalTok{, }\DataTypeTok{contour =}\NormalTok{ T, }\DataTypeTok{lineend=}\StringTok{"square"}\NormalTok{,}\DataTypeTok{linejoin=}\StringTok{"round"}\NormalTok{) }
\NormalTok{  p <-}\StringTok{ }\NormalTok{p }\OperatorTok{+}\StringTok{ }\KeywordTok{geom_point}\NormalTok{(}\DataTypeTok{data=}\NormalTok{d, }\KeywordTok{aes}\NormalTok{(}\DataTypeTok{x =}\NormalTok{ HR, }\DataTypeTok{y =}\NormalTok{ Mass),}
                      \DataTypeTok{color=}\NormalTok{colfunc,}
                      \DataTypeTok{fill=}\NormalTok{colfunc) }\OperatorTok{+}
\StringTok{    }\KeywordTok{scale_color_manual}\NormalTok{(}\DataTypeTok{values =} \KeywordTok{magma}\NormalTok{(}\DecValTok{8}\NormalTok{))}
\NormalTok{  p <-}\StringTok{ }\NormalTok{p }\OperatorTok{+}\StringTok{ }\KeywordTok{scale_y_continuous}\NormalTok{(}\DataTypeTok{limits=}\KeywordTok{c}\NormalTok{(}\OperatorTok{-}\DecValTok{200}\NormalTok{,}\DecValTok{200}\NormalTok{), }\DataTypeTok{name=}\StringTok{"Mass lost (g)"}\NormalTok{) }
\NormalTok{  p <-}\StringTok{ }\NormalTok{p }\OperatorTok{+}\StringTok{ }\KeywordTok{scale_x_continuous}\NormalTok{(}\DataTypeTok{limits=}\KeywordTok{c}\NormalTok{(}\DecValTok{0}\NormalTok{,}\FloatTok{0.35}\NormalTok{),}\DataTypeTok{name=}\KeywordTok{expression}\NormalTok{(}\StringTok{"Home range area (km^2)"}\NormalTok{)) }
\NormalTok{  p <-}\StringTok{ }\NormalTok{p }\OperatorTok{+}\StringTok{ }\KeywordTok{theme_classic}\NormalTok{()}
  \KeywordTok{print}\NormalTok{(p)}
\NormalTok{\}}
\KeywordTok{hr.mass.plot}\NormalTok{(d)}
\end{Highlighting}
\end{Shaded}

Using \texttt{ggplot} when looping through \texttt{for} loop and saving
to dir

\begin{Shaded}
\begin{Highlighting}[]
\KeywordTok{pdf}\NormalTok{(}\StringTok{"mypdf.pdf"}\NormalTok{,}\DataTypeTok{onefile =}\NormalTok{ T)}
\ControlFlowTok{for}\NormalTok{(i }\ControlFlowTok{in} \DecValTok{1}\OperatorTok{:}\DecValTok{3}\NormalTok{)\{ }
\KeywordTok{par}\NormalTok{(}\DataTypeTok{bty=}\StringTok{"n"}\NormalTok{, }\DataTypeTok{las =} \DecValTok{1}\NormalTok{)}
  \KeywordTok{grid.arrange}\NormalTok{( }
  \KeywordTok{ggplot}\NormalTok{(data, }\KeywordTok{aes}\NormalTok{(}\DataTypeTok{x =}\NormalTok{ X, }\DataTypeTok{y =}\NormalTok{ Y, }\DataTypeTok{fill=}\NormalTok{..x..)) }\OperatorTok{+}\StringTok{ }\CommentTok{# geom_density_ridges()}
\StringTok{    }\CommentTok{# scale = overlap}
\StringTok{    }\KeywordTok{geom_density_ridges_gradient}\NormalTok{(}\DataTypeTok{scale =} \DecValTok{5}\NormalTok{, }\DataTypeTok{size=}\FloatTok{0.2}\NormalTok{,}\DataTypeTok{color=}\StringTok{"black"}\NormalTok{, }\DataTypeTok{rel_min_height =} \FloatTok{0.01}\NormalTok{,}\DataTypeTok{panel_scaling=}\NormalTok{T,}\DataTypeTok{alpha=}\FloatTok{0.2}\NormalTok{) }\OperatorTok{+}
\StringTok{    }\KeywordTok{geom_density_ridges}\NormalTok{(}\DataTypeTok{scale =} \DecValTok{5}\NormalTok{, }\DataTypeTok{size=}\FloatTok{0.2}\NormalTok{,}\DataTypeTok{color=}\StringTok{"black"}\NormalTok{, }\DataTypeTok{rel_min_height =} \FloatTok{0.01}\NormalTok{,}\DataTypeTok{fill=}\StringTok{"white"}\NormalTok{,}\DataTypeTok{alpha=}\FloatTok{0.2}\NormalTok{) }\OperatorTok{+}
\StringTok{    }\CommentTok{# geom_density_ridges(scale = 5, size=0.2,color="white", rel_min_height = 0.01,fill=col,alpha=0.5) +}
\StringTok{    }\KeywordTok{scale_fill_viridis}\NormalTok{(}\DataTypeTok{name =} \StringTok{"Diameter"}\NormalTok{, }\DataTypeTok{alpha=}\FloatTok{0.1}\NormalTok{, }\DataTypeTok{option =} \StringTok{"magma"}\NormalTok{,}\DataTypeTok{direction=}\OperatorTok{-}\DecValTok{1}\NormalTok{) }\OperatorTok{+}\StringTok{ }\CommentTok{# "magma", "inferno","plasma", "viridis", "cividis"}
\StringTok{    }\KeywordTok{xlim}\NormalTok{(}\KeywordTok{c}\NormalTok{(}\DecValTok{0}\NormalTok{,}\DecValTok{25}\NormalTok{)) }\OperatorTok{+}
\StringTok{    }\KeywordTok{labs}\NormalTok{(}\DataTypeTok{title =} \KeywordTok{paste0}\NormalTok{(}\StringTok{"Title_"}\NormalTok{,i)) }\OperatorTok{+}
\StringTok{    }\KeywordTok{xlab}\NormalTok{(}\StringTok{"X"}\NormalTok{) }\OperatorTok{+}
\StringTok{    }\KeywordTok{ylab}\NormalTok{(}\StringTok{"Y"}\NormalTok{) }\OperatorTok{+}
\StringTok{    }\CommentTok{# plot_it_gg("white")}
\StringTok{  }\NormalTok{)}
\NormalTok{\} }\CommentTok{# end loop }
\KeywordTok{dev.off}\NormalTok{()}
\end{Highlighting}
\end{Shaded}

\subsubsection{NAs}\label{nas}

Replace NAs with 0's

\begin{Shaded}
\begin{Highlighting}[]
\NormalTok{df[}\KeywordTok{is.na}\NormalTok{(df)] <-}\StringTok{ }\DecValTok{0}
\end{Highlighting}
\end{Shaded}

Replace X values less than given value (V) with 0

\begin{Shaded}
\begin{Highlighting}[]
\NormalTok{df}\OperatorTok{$}\NormalTok{X[df}\OperatorTok{$}\NormalTok{X}\OperatorTok{<}\NormalTok{V] <-}\StringTok{ }\DecValTok{0} 
\end{Highlighting}
\end{Shaded}

Check for NAs

\begin{Shaded}
\begin{Highlighting}[]
\KeywordTok{sapply}\NormalTok{(df, }\ControlFlowTok{function}\NormalTok{(x) }\KeywordTok{sum}\NormalTok{(}\KeywordTok{is.na}\NormalTok{(x)))}
\end{Highlighting}
\end{Shaded}

Replace NaN and Inf values with NA

\begin{Shaded}
\begin{Highlighting}[]
\NormalTok{df}\OperatorTok{$}\NormalTok{col1[}\KeywordTok{which}\NormalTok{(}\OperatorTok{!}\KeywordTok{is.finite}\NormalTok{(df}\OperatorTok{$}\NormalTok{col1))] <-}\StringTok{  }\OtherTok{NA}
\end{Highlighting}
\end{Shaded}

Fill in missing data values in sequence with NA

\begin{Shaded}
\begin{Highlighting}[]
\CommentTok{# /Users/malishev/Documents/Manuscripts/Chapter4/Sims/Chapter4_figs.R}
\KeywordTok{library}\NormalTok{(zoo)}
\NormalTok{data <-}\StringTok{ }\KeywordTok{data.frame}\NormalTok{(}\DataTypeTok{index =} \KeywordTok{c}\NormalTok{(}\DecValTok{1}\OperatorTok{:}\DecValTok{4}\NormalTok{, }\DecValTok{6}\OperatorTok{:}\DecValTok{10}\NormalTok{),}
  \DataTypeTok{data =} \KeywordTok{c}\NormalTok{(}\FloatTok{1.5}\NormalTok{,}\FloatTok{4.3}\NormalTok{,}\FloatTok{5.6}\NormalTok{,}\FloatTok{6.7}\NormalTok{,}\FloatTok{7.1}\NormalTok{,}\FloatTok{12.5}\NormalTok{,}\FloatTok{14.5}\NormalTok{,}\FloatTok{16.8}\NormalTok{,}\FloatTok{3.4}\NormalTok{))}
\CommentTok{#you can create a series}
\NormalTok{z <-}\StringTok{ }\KeywordTok{zoo}\NormalTok{(data}\OperatorTok{$}\NormalTok{data, data}\OperatorTok{$}\NormalTok{index)}
\CommentTok{#end extend it to the grid 1:10}
\NormalTok{z <-}\StringTok{ }\KeywordTok{merge}\NormalTok{(}\KeywordTok{zoo}\NormalTok{(,}\DecValTok{1}\OperatorTok{:}\DecValTok{10}\NormalTok{), z)}

\CommentTok{#worked example}
\CommentTok{# fill in missing Tb values }
\NormalTok{minTb.d <-}\StringTok{ }\KeywordTok{zoo}\NormalTok{(minTb}\OperatorTok{$}\NormalTok{Tick,minTb}\OperatorTok{$}\NormalTok{Days)}
\NormalTok{minTb.d <-}\StringTok{ }\KeywordTok{merge}\NormalTok{(}\KeywordTok{zoo}\NormalTok{(}\OtherTok{NULL}\NormalTok{,}\DecValTok{1}\OperatorTok{:}\NormalTok{days), minTb.d) }\CommentTok{# make the minTb series match the temp series (117 days)}
\NormalTok{minTb.d <-}\StringTok{ }\KeywordTok{as.numeric}\NormalTok{(minTb.d) }\CommentTok{# = time individuals reached VTMIN in ticks}
\NormalTok{minTb <-}\StringTok{ }\NormalTok{minTb.d }\OperatorTok{-}\StringTok{ }\NormalTok{temp}\OperatorTok{$}\NormalTok{Tick }\CommentTok{# get diff between starting time and time to reach VTMIN}
\NormalTok{minTb <-}\StringTok{ }\NormalTok{minTb}\OperatorTok{/}\DecValTok{2} \CommentTok{# convert ticks to minutes}
\NormalTok{minTb <-}\StringTok{ }\NormalTok{minTb}\OperatorTok{/}\DecValTok{60} \CommentTok{#convert to hours}
\NormalTok{minTb <-}\StringTok{ }\KeywordTok{data.frame}\NormalTok{(}\StringTok{"Days"}\NormalTok{=}\DecValTok{1}\OperatorTok{:}\NormalTok{days,}\StringTok{"Time"}\NormalTok{=minTb)}

\CommentTok{# then fill in missing values}
\KeywordTok{approx}\NormalTok{(minTb}\OperatorTok{$}\NormalTok{Time,}\DataTypeTok{method =} \StringTok{"linear"}\NormalTok{)}
\end{Highlighting}
\end{Shaded}

\subsubsection{Plotting}\label{plotting}

Plot one plot window above and two below

\begin{Shaded}
\begin{Highlighting}[]
\KeywordTok{layout}\NormalTok{(}\KeywordTok{matrix}\NormalTok{(}\KeywordTok{c}\NormalTok{(}\DecValTok{1}\NormalTok{,}\DecValTok{1}\NormalTok{,}\DecValTok{2}\NormalTok{,}\DecValTok{3}\NormalTok{), }\DecValTok{2}\NormalTok{, }\DecValTok{2}\NormalTok{, }\DataTypeTok{byrow =} \OtherTok{TRUE}\NormalTok{))}
\end{Highlighting}
\end{Shaded}

Bookend axis ticks for plot E.g. at 0 and 100 when data is 1:99

\begin{Shaded}
\begin{Highlighting}[]
\KeywordTok{axis}\NormalTok{(}\DecValTok{1}\NormalTok{,}\DataTypeTok{at=}\KeywordTok{c}\NormalTok{(}\DecValTok{0}\NormalTok{,}\KeywordTok{length}\NormalTok{(loco}\OperatorTok{$}\NormalTok{X)),}\DataTypeTok{labels=}\KeywordTok{c}\NormalTok{(}\StringTok{""}\NormalTok{,}\StringTok{""}\NormalTok{))}\CommentTok{# bookending axis tick marks}
\end{Highlighting}
\end{Shaded}

Optimal legend formatting for base

\begin{Shaded}
\begin{Highlighting}[]
\KeywordTok{legend}\NormalTok{(}\StringTok{"right"}\NormalTok{,}\DataTypeTok{legend=}\KeywordTok{c}\NormalTok{(}\StringTok{"Small"}\NormalTok{,}\StringTok{"Intermediate"}\NormalTok{,}\StringTok{"Large"}\NormalTok{),}\DataTypeTok{col=}\KeywordTok{c}\NormalTok{(colfunc[colvec[}\DecValTok{1}\OperatorTok{:}\DecValTok{3}\NormalTok{]]),}
       \DataTypeTok{bty=}\StringTok{"n"}\NormalTok{,}\DataTypeTok{pch=}\DecValTok{20}\NormalTok{,}\DataTypeTok{pt.cex=}\FloatTok{1.5}\NormalTok{,}\DataTypeTok{cex=}\FloatTok{0.7}\NormalTok{,}\DataTypeTok{y.intersp =} \FloatTok{0.5}\NormalTok{, }\DataTypeTok{xjust =} \FloatTok{0.5}\NormalTok{,}
       \DataTypeTok{title=}\StringTok{"Size class"}\NormalTok{,}\DataTypeTok{title.adj =} \FloatTok{0.3}\NormalTok{,}\DataTypeTok{text.font=}\DecValTok{2}\NormalTok{,}
       \DataTypeTok{trace=}\NormalTok{T,}\DataTypeTok{inset=}\FloatTok{0.1}\NormalTok{)}
\end{Highlighting}
\end{Shaded}

Plot inset plot in current plot
(\url{https://stackoverflow.com/questions/17041246/how-to-add-an-inset-subplot-to-topright-of-an-r-plot})

\begin{Shaded}
\begin{Highlighting}[]
\CommentTok{# calculate position of inset}
\NormalTok{plotdim <-}\StringTok{ }\KeywordTok{par}\NormalTok{(}\StringTok{"plt"}\NormalTok{)}\CommentTok{# get plot window dims as fraction of current plot dims }
\NormalTok{xleft    =}\StringTok{ }\NormalTok{plotdim[}\DecValTok{2}\NormalTok{] }\OperatorTok{-}\StringTok{ }\NormalTok{(plotdim[}\DecValTok{2}\NormalTok{] }\OperatorTok{-}\StringTok{ }\NormalTok{plotdim[}\DecValTok{1}\NormalTok{]) }\OperatorTok{*}\StringTok{ }\FloatTok{0.5}
\NormalTok{xright   =}\StringTok{ }\NormalTok{plotdim[}\DecValTok{2}\NormalTok{]  }\CommentTok{#}
\NormalTok{ybottom  =}\StringTok{ }\NormalTok{plotdim[}\DecValTok{4}\NormalTok{] }\OperatorTok{-}\StringTok{ }\NormalTok{(plotdim[}\DecValTok{4}\NormalTok{] }\OperatorTok{-}\StringTok{ }\NormalTok{plotdim[}\DecValTok{3}\NormalTok{]) }\OperatorTok{*}\StringTok{ }\FloatTok{0.5}  \CommentTok{#}
\NormalTok{ytop     =}\StringTok{ }\NormalTok{plotdim[}\DecValTok{4}\NormalTok{]  }\CommentTok{#}

\CommentTok{# set position for plot inset}
\KeywordTok{par}\NormalTok{(}\DataTypeTok{fig =} \KeywordTok{c}\NormalTok{(xleft, xright, ybottom, ytop),}\DataTypeTok{mar=}\KeywordTok{c}\NormalTok{(}\DecValTok{0}\NormalTok{,}\DecValTok{0}\NormalTok{,}\DecValTok{0}\NormalTok{,}\DecValTok{0}\NormalTok{),}\DataTypeTok{new=}\OtherTok{TRUE}\NormalTok{)}

\KeywordTok{boxplot}\NormalTok{(Eggs}\OperatorTok{~}\NormalTok{Size,}\DataTypeTok{data=}\NormalTok{meso2,}
                \DataTypeTok{col=}\KeywordTok{adjustcolor}\NormalTok{(colfunc[colvec[}\DecValTok{1}\OperatorTok{:}\DecValTok{3}\NormalTok{]],}\DataTypeTok{alpha=}\FloatTok{0.5}\NormalTok{),}
                \DataTypeTok{notch =}\NormalTok{ T,}\DataTypeTok{xlab=}\StringTok{"Week"}\NormalTok{,}\DataTypeTok{ylab=}\StringTok{"Diameter (mm)"}\NormalTok{,}
                \DataTypeTok{xaxs =} \StringTok{"i"}\NormalTok{, }\DataTypeTok{yaxs =} \StringTok{"i"}
\NormalTok{                ) }
\end{Highlighting}
\end{Shaded}

Interactive plots with rCharts (javascript and d3 viz)\\
\url{http://ramnathv.github.io/rCharts/}

\begin{Shaded}
\begin{Highlighting}[]
\KeywordTok{require}\NormalTok{(devtools)}
\KeywordTok{install_github}\NormalTok{(}\StringTok{'rCharts'}\NormalTok{, }\StringTok{'ramnathv'}\NormalTok{)}
\end{Highlighting}
\end{Shaded}

Cluster plot\\
\url{https://rpubs.com/dgrtwo/technology-clusters}

\begin{Shaded}
\begin{Highlighting}[]
\KeywordTok{library}\NormalTok{(readr)}
\KeywordTok{library}\NormalTok{(dplyr)}
\KeywordTok{library}\NormalTok{(igraph)}
\KeywordTok{library}\NormalTok{(ggraph)}
\KeywordTok{library}\NormalTok{(ggforce)}

\CommentTok{# This shared file contains the number of question that have each pair of tags}
\CommentTok{# This counts only questions that are not deleted and have a positive score}
\NormalTok{tag_pair_data <-}\StringTok{ }\KeywordTok{read_csv}\NormalTok{(}\StringTok{"http://varianceexplained.org/files/tag_pairs.csv.gz"}\NormalTok{)}

\NormalTok{relationships <-}\StringTok{ }\NormalTok{tag_pair_data }\OperatorTok
\StringTok{  }\KeywordTok{mutate}\NormalTok{(}\DataTypeTok{Fraction =}\NormalTok{ Cooccur }\OperatorTok{/}\StringTok{ }\NormalTok{Tag1Total) }\OperatorTok
\StringTok{  }\KeywordTok{filter}\NormalTok{(Fraction }\OperatorTok{>=}\StringTok{ }\NormalTok{.}\DecValTok{35}\NormalTok{) }\OperatorTok
\StringTok{  }\KeywordTok{distinct}\NormalTok{(Tag1)}

\NormalTok{v <-}\StringTok{ }\NormalTok{tag_pair_data }\OperatorTok
\StringTok{  }\KeywordTok{select}\NormalTok{(Tag1, Tag1Total) }\OperatorTok
\StringTok{  }\KeywordTok{distinct}\NormalTok{(Tag1) }\OperatorTok
\StringTok{  }\KeywordTok{filter}\NormalTok{(Tag1 }\OperatorTok\StringTok{ }\NormalTok{relationships}\OperatorTok{$}\NormalTok{Tag1 }\OperatorTok{|}
\StringTok{         }\NormalTok{Tag1 }\OperatorTok\StringTok{ }\NormalTok{relationships}\OperatorTok{$}\NormalTok{Tag2) }\OperatorTok
\StringTok{  }\KeywordTok{arrange}\NormalTok{(}\KeywordTok{desc}\NormalTok{(Tag1Total))}

\NormalTok{a <-}\StringTok{ }\NormalTok{grid}\OperatorTok{::}\KeywordTok{arrow}\NormalTok{(}\DataTypeTok{length =}\NormalTok{ grid}\OperatorTok{::}\KeywordTok{unit}\NormalTok{(.}\DecValTok{08}\NormalTok{, }\StringTok{"inches"}\NormalTok{), }\DataTypeTok{ends =} \StringTok{"first"}\NormalTok{, }\DataTypeTok{type =} \StringTok{"closed"}\NormalTok{)}

\KeywordTok{set.seed}\NormalTok{(}\DecValTok{2016}\NormalTok{)}

\NormalTok{relationships }\OperatorTok
\StringTok{  }\KeywordTok{graph_from_data_frame}\NormalTok{(}\DataTypeTok{vertices =}\NormalTok{ v) }\OperatorTok
\StringTok{  }\KeywordTok{ggraph}\NormalTok{(}\DataTypeTok{layout =} \StringTok{"fr"}\NormalTok{) }\OperatorTok{+}
\StringTok{  }\KeywordTok{geom_edge_link}\NormalTok{(}\KeywordTok{aes}\NormalTok{(}\DataTypeTok{alpha =}\NormalTok{ Fraction), }\DataTypeTok{arrow =}\NormalTok{ a) }\OperatorTok{+}
\StringTok{  }\KeywordTok{geom_node_point}\NormalTok{(}\KeywordTok{aes}\NormalTok{(}\DataTypeTok{size =}\NormalTok{ Tag1Total), }\DataTypeTok{color =} \StringTok{"lightblue"}\NormalTok{) }\OperatorTok{+}
\StringTok{  }\KeywordTok{geom_node_text}\NormalTok{(}\KeywordTok{aes}\NormalTok{(}\DataTypeTok{size =}\NormalTok{ Tag1Total, }\DataTypeTok{label =}\NormalTok{ name), }\DataTypeTok{check_overlap =} \OtherTok{TRUE}\NormalTok{) }\OperatorTok{+}
\StringTok{  }\KeywordTok{scale_size_continuous}\NormalTok{(}\DataTypeTok{range =} \KeywordTok{c}\NormalTok{(}\DecValTok{2}\NormalTok{, }\DecValTok{9}\NormalTok{)) }\OperatorTok{+}
\StringTok{  }\NormalTok{ggforce}\OperatorTok{::}\KeywordTok{theme_no_axes}\NormalTok{() }\OperatorTok{+}
\StringTok{  }\KeywordTok{theme}\NormalTok{(}\DataTypeTok{legend.position =} \StringTok{"none"}\NormalTok{)}
\end{Highlighting}
\end{Shaded}

\subsubsection{Reading in files/data}\label{reading-in-filesdata}

Read in file manually

\begin{Shaded}
\begin{Highlighting}[]
\NormalTok{get.file.vol <-}\StringTok{ }\KeywordTok{read.table}\NormalTok{(}\KeywordTok{file.choose}\NormalTok{())}\CommentTok{#read file manually}
\NormalTok{v.file <-}\StringTok{ }\NormalTok{get.file.vol[}\DecValTok{1}\OperatorTok{:}\DecValTok{100}\NormalTok{,}\DecValTok{1}\NormalTok{]}\CommentTok{#get the volume}
\end{Highlighting}
\end{Shaded}

Loop through files from dir and append to list

\begin{Shaded}
\begin{Highlighting}[]
\CommentTok{# reading in spdf (hrpath) files from drive}
\KeywordTok{setwd}\NormalTok{(}\StringTok{"/Users/camel/Desktop/Matt2016/Manuscripts/MalishevBullKearney/Resubmission/2016/barcoo sims/barcooresults/hrpath_75"}\NormalTok{)}
\NormalTok{file.list<-}\KeywordTok{list.files}\NormalTok{()}
\NormalTok{hrs75<-}\KeywordTok{as.list}\NormalTok{(}\KeywordTok{rep}\NormalTok{(}\DecValTok{1}\NormalTok{,}\DecValTok{100}\NormalTok{)) }\CommentTok{# empty list}
\ControlFlowTok{for}\NormalTok{ (f }\ControlFlowTok{in} \DecValTok{1}\OperatorTok{:}\DecValTok{100}\NormalTok{)\{}
  \KeywordTok{load}\NormalTok{(file.list[f])}
\NormalTok{  hrs75[f]<-hrpath}
\NormalTok{\}}

\CommentTok{# working version}
\CommentTok{#converting spdf into mcp(spdf,100,unout="m2)}
\NormalTok{ghr<-}\KeywordTok{list}\NormalTok{()}
\ControlFlowTok{for}\NormalTok{ (i }\ControlFlowTok{in}\NormalTok{ hrs75[}\DecValTok{1}\OperatorTok{:}\DecValTok{10}\NormalTok{]) \{}
\NormalTok{  m<-}\KeywordTok{mcp}\NormalTok{(i,}\DecValTok{100}\NormalTok{,}\DataTypeTok{unout=}\StringTok{'m2'}\NormalTok{)}
\NormalTok{  ghr<-}\KeywordTok{c}\NormalTok{(ghr,m)}
\NormalTok{\};ghr}
\end{Highlighting}
\end{Shaded}

\subsubsection{Subsetting}\label{subsetting}

Select specific rows E.g. select rows of sfeed\_move not in foodh

\begin{Shaded}
\begin{Highlighting}[]
\KeywordTok{library}\NormalTok{(sqldf)}
\NormalTok{a1NotIna2_h  <-}\StringTok{ }\KeywordTok{sqldf}\NormalTok{(}\StringTok{'SELECT * FROM sfeed_move EXCEPT SELECT * FROM foodh'}\NormalTok{)}
\NormalTok{a1NotIna2_l  <-}\StringTok{ }\KeywordTok{sqldf}\NormalTok{(}\StringTok{'SELECT * FROM sfeed_move EXCEPT SELECT * FROM foodl'}\NormalTok{)}
\CommentTok{# select rows from sfeed_move that also appear in foodh}
\NormalTok{a1Ina2_h  <-}\StringTok{ }\KeywordTok{sqldf}\NormalTok{(}\StringTok{'SELECT * FROM sfeed_move INTERSECT SELECT * FROM foodh'}\NormalTok{)}
\NormalTok{a1Ina2_l  <-}\StringTok{ }\KeywordTok{sqldf}\NormalTok{(}\StringTok{'SELECT * FROM sfeed_move INTERSECT SELECT * FROM foodl'}\NormalTok{)}
\end{Highlighting}
\end{Shaded}

\subparagraph{}\label{section}

\subsubsection{R Markdown}\label{r-markdown}

Hide unwanted code output, such as inherent examples for functions

\begin{Shaded}
\begin{Highlighting}[]
\CommentTok{# ```\{r, cache = TRUE, tidy = TRUE, lazy = TRUE, results='markup'\}}
\end{Highlighting}
\end{Shaded}

\subsubsection{Web scraping}\label{web-scraping}

Scraping web tables\\
\url{http://web.mit.edu/~r/current/arch/i386_linux26/lib/R/library/XML/html/readHTMLTable.html\%5Bhttp://web.mit.edu/~r/current/arch/i386_linux26/lib/R/library/XML/html/readHTMLTable.html\%5D}

\begin{Shaded}
\begin{Highlighting}[]
\KeywordTok{library}\NormalTok{(XML)}
\KeywordTok{readHTMLTable}\NormalTok{()}
\end{Highlighting}
\end{Shaded}

Scraping Twitter timelines\\
See complete example at
\url{http://varianceexplained.org/r/trump-tweets/}

\begin{Shaded}
\begin{Highlighting}[]
\CommentTok{# https://cran.r-project.org/web/packages/twitteR/}
\KeywordTok{library}\NormalTok{(dplyr)}
\KeywordTok{library}\NormalTok{(purrr)}
\KeywordTok{library}\NormalTok{(twitteR)}
\end{Highlighting}
\end{Shaded}

\printbibliography

\end{document}
