\documentclass[10,portrait]{article}
\usepackage{lmodern}
\usepackage{amssymb,amsmath}
\usepackage{ifxetex,ifluatex}
\usepackage{fixltx2e} % provides \textsubscript
\ifnum 0\ifxetex 1\fi\ifluatex 1\fi=0 % if pdftex
  \usepackage[T1]{fontenc}
  \usepackage[utf8]{inputenc}
\else % if luatex or xelatex
  \ifxetex
    \usepackage{mathspec}
  \else
    \usepackage{fontspec}
  \fi
  \defaultfontfeatures{Ligatures=TeX,Scale=MatchLowercase}
\fi
% use upquote if available, for straight quotes in verbatim environments
\IfFileExists{upquote.sty}{\usepackage{upquote}}{}
% use microtype if available
\IfFileExists{microtype.sty}{%
\usepackage[]{microtype}
\UseMicrotypeSet[protrusion]{basicmath} % disable protrusion for tt fonts
}{}
\PassOptionsToPackage{hyphens}{url} % url is loaded by hyperref
\usepackage[unicode=true]{hyperref}
\PassOptionsToPackage{usenames,dvipsnames}{color} % color is loaded by hyperref
\hypersetup{
            pdftitle={Useful R code},
            colorlinks=true,
            linkcolor=pink,
            citecolor=red,
            urlcolor=blue,
            breaklinks=true}
\urlstyle{same}  % don't use monospace font for urls
\usepackage[margin=1in]{geometry}
\usepackage[]{biblatex}
\usepackage{color}
\usepackage{fancyvrb}
\newcommand{\VerbBar}{|}
\newcommand{\VERB}{\Verb[commandchars=\\\{\}]}
\DefineVerbatimEnvironment{Highlighting}{Verbatim}{commandchars=\\\{\}}
% Add ',fontsize=\small' for more characters per line
\usepackage{framed}
\definecolor{shadecolor}{RGB}{248,248,248}
\newenvironment{Shaded}{\begin{snugshade}}{\end{snugshade}}
\newcommand{\KeywordTok}[1]{\textcolor[rgb]{0.13,0.29,0.53}{\textbf{#1}}}
\newcommand{\DataTypeTok}[1]{\textcolor[rgb]{0.13,0.29,0.53}{#1}}
\newcommand{\DecValTok}[1]{\textcolor[rgb]{0.00,0.00,0.81}{#1}}
\newcommand{\BaseNTok}[1]{\textcolor[rgb]{0.00,0.00,0.81}{#1}}
\newcommand{\FloatTok}[1]{\textcolor[rgb]{0.00,0.00,0.81}{#1}}
\newcommand{\ConstantTok}[1]{\textcolor[rgb]{0.00,0.00,0.00}{#1}}
\newcommand{\CharTok}[1]{\textcolor[rgb]{0.31,0.60,0.02}{#1}}
\newcommand{\SpecialCharTok}[1]{\textcolor[rgb]{0.00,0.00,0.00}{#1}}
\newcommand{\StringTok}[1]{\textcolor[rgb]{0.31,0.60,0.02}{#1}}
\newcommand{\VerbatimStringTok}[1]{\textcolor[rgb]{0.31,0.60,0.02}{#1}}
\newcommand{\SpecialStringTok}[1]{\textcolor[rgb]{0.31,0.60,0.02}{#1}}
\newcommand{\ImportTok}[1]{#1}
\newcommand{\CommentTok}[1]{\textcolor[rgb]{0.56,0.35,0.01}{\textit{#1}}}
\newcommand{\DocumentationTok}[1]{\textcolor[rgb]{0.56,0.35,0.01}{\textbf{\textit{#1}}}}
\newcommand{\AnnotationTok}[1]{\textcolor[rgb]{0.56,0.35,0.01}{\textbf{\textit{#1}}}}
\newcommand{\CommentVarTok}[1]{\textcolor[rgb]{0.56,0.35,0.01}{\textbf{\textit{#1}}}}
\newcommand{\OtherTok}[1]{\textcolor[rgb]{0.56,0.35,0.01}{#1}}
\newcommand{\FunctionTok}[1]{\textcolor[rgb]{0.00,0.00,0.00}{#1}}
\newcommand{\VariableTok}[1]{\textcolor[rgb]{0.00,0.00,0.00}{#1}}
\newcommand{\ControlFlowTok}[1]{\textcolor[rgb]{0.13,0.29,0.53}{\textbf{#1}}}
\newcommand{\OperatorTok}[1]{\textcolor[rgb]{0.81,0.36,0.00}{\textbf{#1}}}
\newcommand{\BuiltInTok}[1]{#1}
\newcommand{\ExtensionTok}[1]{#1}
\newcommand{\PreprocessorTok}[1]{\textcolor[rgb]{0.56,0.35,0.01}{\textit{#1}}}
\newcommand{\AttributeTok}[1]{\textcolor[rgb]{0.77,0.63,0.00}{#1}}
\newcommand{\RegionMarkerTok}[1]{#1}
\newcommand{\InformationTok}[1]{\textcolor[rgb]{0.56,0.35,0.01}{\textbf{\textit{#1}}}}
\newcommand{\WarningTok}[1]{\textcolor[rgb]{0.56,0.35,0.01}{\textbf{\textit{#1}}}}
\newcommand{\AlertTok}[1]{\textcolor[rgb]{0.94,0.16,0.16}{#1}}
\newcommand{\ErrorTok}[1]{\textcolor[rgb]{0.64,0.00,0.00}{\textbf{#1}}}
\newcommand{\NormalTok}[1]{#1}
\IfFileExists{parskip.sty}{%
\usepackage{parskip}
}{% else
\setlength{\parindent}{0pt}
\setlength{\parskip}{6pt plus 2pt minus 1pt}
}
\setlength{\emergencystretch}{3em}  % prevent overfull lines
\providecommand{\tightlist}{%
  \setlength{\itemsep}{0pt}\setlength{\parskip}{0pt}}
\setcounter{secnumdepth}{0}
% Redefines (sub)paragraphs to behave more like sections
\ifx\paragraph\undefined\else
\let\oldparagraph\paragraph
\renewcommand{\paragraph}[1]{\oldparagraph{#1}\mbox{}}
\fi
\ifx\subparagraph\undefined\else
\let\oldsubparagraph\subparagraph
\renewcommand{\subparagraph}[1]{\oldsubparagraph{#1}\mbox{}}
\fi

% set default figure placement to htbp
\makeatletter
\def\fps@figure{htbp}
\makeatother


\title{Useful R code}
\author{Matthew
Malishev\textsuperscript{1}*\\[2\baselineskip]\emph{\textsuperscript{1}
Department of Biology, Emory University, 1510 Clifton Road NE, Atlanta,
GA, USA, 30322}}
\date{}

\begin{document}
\maketitle

{
\hypersetup{linkcolor=black}
\setcounter{tocdepth}{4}
\tableofcontents
}
\begin{verbatim}
## Warning: package 'rmarkdown' was built under R version 3.5.2
\end{verbatim}

~

Date: 2019-05-29\\
\texttt{R} version: 3.5.0\\
*Corresponding author:
\href{mailto:matthew.malishev@gmail.com}{\nolinkurl{matthew.malishev@gmail.com}}\\
This document can be found at
\url{https://github.com/darwinanddavis/UsefulCode}

\newpage  

\subsection{Overview}\label{overview}

This document outlines some useful \texttt{R} code for plotting, cool
functions, and other random tidbits.

\subsubsection{Install dependencies}\label{install-dependencies}

\subsubsection{Attributes}\label{attributes}

Access structural attributes of unique classes, such as raster and ggmap
(bbox).

\begin{Shaded}
\begin{Highlighting}[]
\CommentTok{# Normal example}
\NormalTok{df <-}\StringTok{ }\KeywordTok{data.frame}\NormalTok{(}\StringTok{"X"}\NormalTok{=}\KeywordTok{c}\NormalTok{(}\DecValTok{1}\OperatorTok{:}\DecValTok{5}\NormalTok{),}\StringTok{"Y"}\NormalTok{=}\KeywordTok{c}\NormalTok{(}\DecValTok{6}\OperatorTok{:}\DecValTok{10}\NormalTok{))}
\KeywordTok{str}\NormalTok{(df)}
\NormalTok{df}\OperatorTok{$}\NormalTok{X}

\CommentTok{# `attr` method}
\KeywordTok{require}\NormalTok{(ggmap)}
\NormalTok{map <-}\StringTok{ }\KeywordTok{get_map}\NormalTok{(}\StringTok{"Atlanta"}\NormalTok{,}\DataTypeTok{zoom=}\DecValTok{12}\NormalTok{,}\DataTypeTok{source=}\StringTok{"stamen"}\NormalTok{,}\DataTypeTok{maptype=}\StringTok{"toner-lines"}\NormalTok{)}
\KeywordTok{str}\NormalTok{(map)}
\KeywordTok{attr}\NormalTok{(map,}\StringTok{"bb"}\NormalTok{)}\OperatorTok{$}\NormalTok{ll.lat}
\end{Highlighting}
\end{Shaded}

\subsubsection{Classes}\label{classes}

Convert character to factor to numeric without conversion error

\begin{Shaded}
\begin{Highlighting}[]
\KeywordTok{read.table}\NormalTok{(f,}\DataTypeTok{header=}\NormalTok{T,}\DataTypeTok{sep=}\StringTok{","}\NormalTok{,}\DataTypeTok{row.names=}\OtherTok{NULL}\NormalTok{,}\DataTypeTok{stringsAsFactors=}\OtherTok{FALSE}\NormalTok{, }\DataTypeTok{strip.white=}\OtherTok{TRUE}\NormalTok{)}
\NormalTok{f}\OperatorTok{$}\NormalTok{V2<-}\KeywordTok{as.numeric}\NormalTok{(f}\OperatorTok{$}\NormalTok{V2)}
\end{Highlighting}
\end{Shaded}

See call options for class

\begin{Shaded}
\begin{Highlighting}[]
\KeywordTok{methods}\NormalTok{(}\DataTypeTok{class=}\StringTok{"estUDm"}\NormalTok{)}
\end{Highlighting}
\end{Shaded}

Set dynamic input for variable / assign variable to char vector

\begin{Shaded}
\begin{Highlighting}[]
\NormalTok{shadedens<-}\ControlFlowTok{function}\NormalTok{(shadedens)\{ }\CommentTok{# set shade density to clumped (to match food) or sparse }
  \ControlFlowTok{if}\NormalTok{ (shadedens }\OperatorTok{==}\StringTok{ "Random"}\NormalTok{)\{}
    \KeywordTok{NLCommand}\NormalTok{(}\StringTok{"set Shade-density }\CharTok{\textbackslash{}"}\StringTok{Random}\CharTok{\textbackslash{}"}\StringTok{ "}\NormalTok{) }
\NormalTok{    \}}\ControlFlowTok{else}\NormalTok{\{}
    \KeywordTok{NLCommand}\NormalTok{(}\StringTok{"set Shade-density }\CharTok{\textbackslash{}"}\StringTok{Clumped}\CharTok{\textbackslash{}"}\StringTok{ "}\NormalTok{) }
\NormalTok{    \}}
\NormalTok{  \}}
\KeywordTok{shadedens}\NormalTok{(}\StringTok{"Clumped"}\NormalTok{) }\CommentTok{# set clumped resources}
\end{Highlighting}
\end{Shaded}

\subsubsection{D3 apps}\label{d3-apps}

Interactive network plots using d3

\begin{Shaded}
\begin{Highlighting}[]
\CommentTok{# Load package}
\KeywordTok{install.packages}\NormalTok{(}\StringTok{"networkD3"}\NormalTok{)}
\KeywordTok{library}\NormalTok{(networkD3)}

\CommentTok{# Load energy projection data}
\NormalTok{URL <-}\StringTok{ "https://cdn.rawgit.com/christophergandrud/networkD3/master/JSONdata/energy.json"}
\NormalTok{Energy <-}\StringTok{ }\NormalTok{jsonlite}\OperatorTok{::}\KeywordTok{fromJSON}\NormalTok{(URL)}

\CommentTok{# Now we have 2 data frames: a 'links' data frame with 3 columns (from, to, value), and a 'nodes' data frame that gives the name of each node.}
\KeywordTok{head}\NormalTok{(Energy}\OperatorTok{$}\NormalTok{links)}
\KeywordTok{head}\NormalTok{(Energy}\OperatorTok{$}\NormalTok{nodes)}

\CommentTok{# Thus we can plot it}
\KeywordTok{sankeyNetwork}\NormalTok{(}\DataTypeTok{Links =}\NormalTok{ Energy}\OperatorTok{$}\NormalTok{links, }\DataTypeTok{Nodes =}\NormalTok{ Energy}\OperatorTok{$}\NormalTok{nodes, }\DataTypeTok{Source =} \StringTok{"source"}\NormalTok{,}
              \DataTypeTok{Target =} \StringTok{"target"}\NormalTok{, }\DataTypeTok{Value =} \StringTok{"value"}\NormalTok{, }\DataTypeTok{NodeID =} \StringTok{"name"}\NormalTok{,}
              \DataTypeTok{units =} \StringTok{"TWh"}\NormalTok{, }\DataTypeTok{fontSize =} \DecValTok{12}\NormalTok{, }\DataTypeTok{nodeWidth =} \DecValTok{30}\NormalTok{)}

\NormalTok{?sankeyNetwork}
\end{Highlighting}
\end{Shaded}

\subsubsection{Dataframes}\label{dataframes}

Optimal empty data frame

\begin{Shaded}
\begin{Highlighting}[]
\NormalTok{df <-}\StringTok{ }\KeywordTok{data.frame}\NormalTok{(}\DataTypeTok{Date=}\KeywordTok{as.Date}\NormalTok{(}\KeywordTok{character}\NormalTok{()),}
                 \DataTypeTok{X=}\KeywordTok{numeric}\NormalTok{(), }
                 \DataTypeTok{Y=}\KeywordTok{integer}\NormalTok{(), }
                 \DataTypeTok{stringsAsFactors=}\OtherTok{FALSE}\NormalTok{) }
\end{Highlighting}
\end{Shaded}

Add df cols with \texttt{mutate}

\begin{Shaded}
\begin{Highlighting}[]
\KeywordTok{require}\NormalTok{(dplyr)  }
\NormalTok{df <-}\StringTok{ }\KeywordTok{data.frame}\NormalTok{(}\StringTok{"a"}\NormalTok{=}\KeywordTok{rnorm}\NormalTok{(}\DecValTok{10}\NormalTok{),}\StringTok{"b"}\NormalTok{=(}\DecValTok{1}\OperatorTok{:}\DecValTok{20}\NormalTok{))}
\NormalTok{df }\OperatorTok\StringTok{ }
\StringTok{  }\KeywordTok{mutate}\NormalTok{(}
  \StringTok{"c"}\NormalTok{=}\KeywordTok{rnorm}\NormalTok{(}\DecValTok{20}\NormalTok{),}
  \DataTypeTok{b =}\NormalTok{ b }\OperatorTok{*}\DecValTok{67}
\NormalTok{  )}
\end{Highlighting}
\end{Shaded}

Change \texttt{df} column names

\begin{Shaded}
\begin{Highlighting}[]
\KeywordTok{colnames}\NormalTok{(data)[}\KeywordTok{c}\NormalTok{(}\DecValTok{1}\NormalTok{,}\DecValTok{2}\NormalTok{,}\DecValTok{3}\NormalTok{)] <-}\StringTok{ }\KeywordTok{c}\NormalTok{(}\StringTok{"TimeStamp"}\NormalTok{,}\StringTok{"Lat"}\NormalTok{,}\StringTok{"Long"}\NormalTok{)}
\end{Highlighting}
\end{Shaded}

Remove multiple columns from df

\begin{Shaded}
\begin{Highlighting}[]
\NormalTok{### Remove multiple NA columns  }
\NormalTok{rm_cols <-}\StringTok{ }\KeywordTok{grep}\NormalTok{(}\StringTok{"NA"}\NormalTok{, }\KeywordTok{names}\NormalTok{(tt), }\DataTypeTok{ignore.case =}\NormalTok{ F)}
\NormalTok{df[,}\KeywordTok{colnames}\NormalTok{(df[,rm_cols])] <-}\StringTok{ }\KeywordTok{list}\NormalTok{(}\OtherTok{NULL}\NormalTok{)}
\end{Highlighting}
\end{Shaded}

Check number of characters in each column

\begin{Shaded}
\begin{Highlighting}[]
\KeywordTok{sapply}\NormalTok{(meso1, }\ControlFlowTok{function}\NormalTok{(x) }\KeywordTok{sum}\NormalTok{(}\KeywordTok{nchar}\NormalTok{(x)))   }
\end{Highlighting}
\end{Shaded}

\subsubsection{Generic functions}\label{generic-functions}

Generic useful functions that I can't place under any other headings
here

\begin{Shaded}
\begin{Highlighting}[]
\CommentTok{# dput() for converting outputs such as copied text or data tables into vectors}
\NormalTok{xx <-}\StringTok{ "Some copied text or table from the internet"}
\KeywordTok{dput}\NormalTok{(xx)}
\end{Highlighting}
\end{Shaded}

Round up integers to optimal rounded value

\begin{Shaded}
\begin{Highlighting}[]
\NormalTok{nn <-}\StringTok{ }\KeywordTok{c}\NormalTok{(}\DecValTok{46}\NormalTok{,}\DecValTok{11}\NormalTok{,}\DecValTok{23}\NormalTok{)}
\KeywordTok{round_any}\NormalTok{(nn,}\DecValTok{10}\NormalTok{)}
\KeywordTok{round_any}\NormalTok{(nn,}\DecValTok{10}\NormalTok{,ceiling)}
\KeywordTok{round_any}\NormalTok{(nn,}\DecValTok{10}\NormalTok{,floor)}
\end{Highlighting}
\end{Shaded}

Get summary stats for dataset (means)

\begin{Shaded}
\begin{Highlighting}[]
\NormalTok{means =}\StringTok{ }\KeywordTok{aggregate}\NormalTok{(Cumulative_cercs }\OperatorTok{~}\StringTok{ }\NormalTok{r}\OperatorTok{*}\NormalTok{hb, }\DataTypeTok{data=}\NormalTok{df,  }\DataTypeTok{FUN=}\NormalTok{mean)}
\end{Highlighting}
\end{Shaded}

\subsubsection{\texorpdfstring{\texttt{ggplot}
functions}{ggplot functions}}\label{ggplot-functions}

Remove annoying stock gridlines from plot window

\begin{Shaded}
\begin{Highlighting}[]
\NormalTok{plot }\OperatorTok{+}\StringTok{ }\KeywordTok{theme_bw}\NormalTok{() }\OperatorTok{+}\StringTok{ }
\StringTok{  }\KeywordTok{theme}\NormalTok{(}\DataTypeTok{panel.border =} \KeywordTok{element_blank}\NormalTok{(), }\DataTypeTok{panel.grid.major =} \KeywordTok{element_blank}\NormalTok{(),}
                            \DataTypeTok{panel.grid.minor =} \KeywordTok{element_blank}\NormalTok{(), }\DataTypeTok{axis.line =} \KeywordTok{element_line}\NormalTok{(}\DataTypeTok{colour =} \StringTok{"black"}\NormalTok{))}
\CommentTok{# alternative (after loading ggridges library)}
\KeywordTok{theme_ridges}\NormalTok{(}\DataTypeTok{grid=}\NormalTok{F,}\DataTypeTok{center_axis_labels =}\NormalTok{ T)}
\end{Highlighting}
\end{Shaded}

Setting global graphics theme for ggplot

\begin{Shaded}
\begin{Highlighting}[]
\NormalTok{plot_it_gg <-}\StringTok{ }\ControlFlowTok{function}\NormalTok{(bg,family)\{ }\CommentTok{# bg = colour to plot bg, family = font family}
  \KeywordTok{theme_tufte}\NormalTok{(}\DataTypeTok{base_family =}\NormalTok{ family) }\OperatorTok{+}
\StringTok{  }\KeywordTok{theme}\NormalTok{(}\DataTypeTok{panel.border =} \KeywordTok{element_blank}\NormalTok{(),}
        \DataTypeTok{panel.grid.major =} \KeywordTok{element_blank}\NormalTok{(),}
        \DataTypeTok{panel.grid.minor =} \KeywordTok{element_blank}\NormalTok{(),}
        \DataTypeTok{panel.background =} \KeywordTok{element_rect}\NormalTok{(}\DataTypeTok{fill =}\NormalTok{ bg,}
                                        \DataTypeTok{colour =}\NormalTok{ bg),}
        \DataTypeTok{plot.background =} \KeywordTok{element_rect}\NormalTok{(}\DataTypeTok{fill=}\NormalTok{bg)}
\NormalTok{  ) }\OperatorTok{+}
\StringTok{    }\KeywordTok{theme}\NormalTok{(}\DataTypeTok{axis.line =} \KeywordTok{element_line}\NormalTok{(}\DataTypeTok{color =} \StringTok{"white"}\NormalTok{)) }\OperatorTok{+}
\StringTok{    }\KeywordTok{theme}\NormalTok{(}\DataTypeTok{axis.ticks =} \KeywordTok{element_line}\NormalTok{(}\DataTypeTok{color =} \StringTok{"white"}\NormalTok{)) }\OperatorTok{+}
\StringTok{    }\KeywordTok{theme}\NormalTok{(}\DataTypeTok{plot.title =} \KeywordTok{element_text}\NormalTok{(}\DataTypeTok{colour =} \StringTok{"white"}\NormalTok{)) }\OperatorTok{+}
\StringTok{    }\KeywordTok{theme}\NormalTok{(}\DataTypeTok{axis.title.x =} \KeywordTok{element_text}\NormalTok{(}\DataTypeTok{colour =} \StringTok{"white"}\NormalTok{), }
          \DataTypeTok{axis.title.y =} \KeywordTok{element_text}\NormalTok{(}\DataTypeTok{colour =} \StringTok{"white"}\NormalTok{)) }\OperatorTok{+}
\StringTok{    }\KeywordTok{theme}\NormalTok{(}\DataTypeTok{axis.text.x =} \KeywordTok{element_text}\NormalTok{(}\DataTypeTok{color =} \StringTok{"white"}\NormalTok{),}
          \DataTypeTok{axis.text.y =} \KeywordTok{element_text}\NormalTok{(}\DataTypeTok{color =} \StringTok{"white"}\NormalTok{)) }\OperatorTok{+}
\StringTok{    }\KeywordTok{theme}\NormalTok{(}\DataTypeTok{legend.key =} \KeywordTok{element_rect}\NormalTok{(}\DataTypeTok{fill =}\NormalTok{ bg)) }\OperatorTok{+}\StringTok{ }\CommentTok{# fill bg of legend}
\StringTok{    }\KeywordTok{theme}\NormalTok{(}\DataTypeTok{legend.title =} \KeywordTok{element_text}\NormalTok{(}\DataTypeTok{colour=}\StringTok{"white"}\NormalTok{)) }\OperatorTok{+}\StringTok{ }\CommentTok{# legend title}
\StringTok{    }\KeywordTok{theme}\NormalTok{(}\DataTypeTok{legend.text =} \KeywordTok{element_text}\NormalTok{(}\DataTypeTok{colour=}\StringTok{"white"}\NormalTok{)) }\CommentTok{# legend labels}
\NormalTok{\} }
\end{Highlighting}
\end{Shaded}

Put plot in function to take dynamic data inputs\\
Ref:
\url{http://jcborras.net/carpet/visualizing-political-divergences-2012-local-elections-in-helsinki.html}

\begin{Shaded}
\begin{Highlighting}[]
\NormalTok{hr.mass.plot <-}\StringTok{ }\ControlFlowTok{function}\NormalTok{(d) \{}
\NormalTok{  p <-}\StringTok{ }\KeywordTok{ggplot}\NormalTok{(d, }\KeywordTok{aes}\NormalTok{(HR, Mass, }\DataTypeTok{color =}\NormalTok{ colfunc)) }\OperatorTok{+}\StringTok{ }
\StringTok{    }\KeywordTok{geom_density_2d}\NormalTok{(}\DataTypeTok{data=}\NormalTok{d, }\KeywordTok{aes}\NormalTok{(}\DataTypeTok{x =}\NormalTok{ HR, }\DataTypeTok{y =}\NormalTok{ Mass), }
                    \DataTypeTok{stat =} \StringTok{"density2d"}\NormalTok{,}\DataTypeTok{position=}\StringTok{"identity"}\NormalTok{, }
                    \DataTypeTok{color=}\KeywordTok{adjustcolor}\NormalTok{(}\StringTok{"orange"}\NormalTok{,}\DataTypeTok{alpha=}\FloatTok{0.8}\NormalTok{), }\DataTypeTok{size=}\FloatTok{1.5}\NormalTok{, }\DataTypeTok{contour =}\NormalTok{ T, }\DataTypeTok{lineend=}\StringTok{"square"}\NormalTok{,}\DataTypeTok{linejoin=}\StringTok{"round"}\NormalTok{) }
\NormalTok{  p <-}\StringTok{ }\NormalTok{p }\OperatorTok{+}\StringTok{ }\KeywordTok{geom_point}\NormalTok{(}\DataTypeTok{data=}\NormalTok{d, }\KeywordTok{aes}\NormalTok{(}\DataTypeTok{x =}\NormalTok{ HR, }\DataTypeTok{y =}\NormalTok{ Mass),}
                      \DataTypeTok{color=}\NormalTok{colfunc,}
                      \DataTypeTok{fill=}\NormalTok{colfunc) }\OperatorTok{+}
\StringTok{    }\KeywordTok{scale_color_manual}\NormalTok{(}\DataTypeTok{values =} \KeywordTok{magma}\NormalTok{(}\DecValTok{8}\NormalTok{))}
\NormalTok{  p <-}\StringTok{ }\NormalTok{p }\OperatorTok{+}\StringTok{ }\KeywordTok{scale_y_continuous}\NormalTok{(}\DataTypeTok{limits=}\KeywordTok{c}\NormalTok{(}\OperatorTok{-}\DecValTok{200}\NormalTok{,}\DecValTok{200}\NormalTok{), }\DataTypeTok{name=}\StringTok{"Mass lost (g)"}\NormalTok{) }
\NormalTok{  p <-}\StringTok{ }\NormalTok{p }\OperatorTok{+}\StringTok{ }\KeywordTok{scale_x_continuous}\NormalTok{(}\DataTypeTok{limits=}\KeywordTok{c}\NormalTok{(}\DecValTok{0}\NormalTok{,}\FloatTok{0.35}\NormalTok{),}\DataTypeTok{name=}\KeywordTok{expression}\NormalTok{(}\StringTok{"Home range area (km^2)"}\NormalTok{)) }
\NormalTok{  p <-}\StringTok{ }\NormalTok{p }\OperatorTok{+}\StringTok{ }\KeywordTok{theme_classic}\NormalTok{()}
  \KeywordTok{print}\NormalTok{(p)}
\NormalTok{\}}
\KeywordTok{hr.mass.plot}\NormalTok{(d)}
\end{Highlighting}
\end{Shaded}

Using \texttt{ggplot} when looping through \texttt{for} loop and saving
to dir

\begin{Shaded}
\begin{Highlighting}[]
\KeywordTok{pdf}\NormalTok{(}\StringTok{"mypdf.pdf"}\NormalTok{,}\DataTypeTok{onefile =}\NormalTok{ T)}
\ControlFlowTok{for}\NormalTok{(i }\ControlFlowTok{in} \DecValTok{1}\OperatorTok{:}\DecValTok{3}\NormalTok{)\{ }
\KeywordTok{par}\NormalTok{(}\DataTypeTok{bty=}\StringTok{"n"}\NormalTok{, }\DataTypeTok{las =} \DecValTok{1}\NormalTok{)}
  \KeywordTok{grid.arrange}\NormalTok{( }
  \KeywordTok{ggplot}\NormalTok{(data, }\KeywordTok{aes}\NormalTok{(}\DataTypeTok{x =}\NormalTok{ X, }\DataTypeTok{y =}\NormalTok{ Y, }\DataTypeTok{fill=}\NormalTok{..x..)) }\OperatorTok{+}\StringTok{ }\CommentTok{# geom_density_ridges()}
\StringTok{    }\CommentTok{# scale = overlap}
\StringTok{    }\KeywordTok{geom_density_ridges_gradient}\NormalTok{(}\DataTypeTok{scale =} \DecValTok{5}\NormalTok{, }\DataTypeTok{size=}\FloatTok{0.2}\NormalTok{,}\DataTypeTok{color=}\StringTok{"black"}\NormalTok{, }\DataTypeTok{rel_min_height =} \FloatTok{0.01}\NormalTok{,}\DataTypeTok{panel_scaling=}\NormalTok{T,}\DataTypeTok{alpha=}\FloatTok{0.2}\NormalTok{) }\OperatorTok{+}
\StringTok{    }\KeywordTok{geom_density_ridges}\NormalTok{(}\DataTypeTok{scale =} \DecValTok{5}\NormalTok{, }\DataTypeTok{size=}\FloatTok{0.2}\NormalTok{,}\DataTypeTok{color=}\StringTok{"black"}\NormalTok{, }\DataTypeTok{rel_min_height =} \FloatTok{0.01}\NormalTok{,}\DataTypeTok{fill=}\StringTok{"white"}\NormalTok{,}\DataTypeTok{alpha=}\FloatTok{0.2}\NormalTok{) }\OperatorTok{+}
\StringTok{    }\CommentTok{# geom_density_ridges(scale = 5, size=0.2,color="white", rel_min_height = 0.01,fill=col,alpha=0.5) +}
\StringTok{    }\KeywordTok{scale_fill_viridis}\NormalTok{(}\DataTypeTok{name =} \StringTok{"Diameter"}\NormalTok{, }\DataTypeTok{alpha=}\FloatTok{0.1}\NormalTok{, }\DataTypeTok{option =} \StringTok{"magma"}\NormalTok{,}\DataTypeTok{direction=}\OperatorTok{-}\DecValTok{1}\NormalTok{) }\OperatorTok{+}\StringTok{ }\CommentTok{# "magma", "inferno","plasma", "viridis", "cividis"}
\StringTok{    }\KeywordTok{xlim}\NormalTok{(}\KeywordTok{c}\NormalTok{(}\DecValTok{0}\NormalTok{,}\DecValTok{25}\NormalTok{)) }\OperatorTok{+}
\StringTok{    }\KeywordTok{labs}\NormalTok{(}\DataTypeTok{title =} \KeywordTok{paste0}\NormalTok{(}\StringTok{"Title_"}\NormalTok{,i)) }\OperatorTok{+}
\StringTok{    }\KeywordTok{xlab}\NormalTok{(}\StringTok{"X"}\NormalTok{) }\OperatorTok{+}
\StringTok{    }\KeywordTok{ylab}\NormalTok{(}\StringTok{"Y"}\NormalTok{) }\OperatorTok{+}
\StringTok{    }\CommentTok{# plot_it_gg("white")}
\StringTok{  }\NormalTok{)}
\NormalTok{\} }\CommentTok{# end loop }
\KeywordTok{dev.off}\NormalTok{()}
\end{Highlighting}
\end{Shaded}

Converting lists and dataframes to usable format for \texttt{ggplot}
(\texttt{melt} package)

\begin{Shaded}
\begin{Highlighting}[]
\CommentTok{# ------------------------- plot individual outputs -------------------------}
\NormalTok{mm_ =}\StringTok{ }\KeywordTok{readRDS}\NormalTok{(}\KeywordTok{paste0}\NormalTok{(model.path,fh,}\StringTok{".R"}\NormalTok{))}
\KeywordTok{cat}\NormalTok{(}\StringTok{"order = cerc, food, juv, adult, infected, infected shedding, host length, parasite mass"}\NormalTok{)}
\CommentTok{# plot master}
\NormalTok{mm <-}\StringTok{ }\NormalTok{mm_[[}\DecValTok{2}\NormalTok{]]}
\NormalTok{y_m <-}\StringTok{ }\KeywordTok{melt}\NormalTok{(mm);y_m}
\KeywordTok{ggplot}\NormalTok{() }\OperatorTok{+}
\StringTok{  }\KeywordTok{geom_point}\NormalTok{(}\DataTypeTok{data =}\NormalTok{ y_m, }\KeywordTok{aes}\NormalTok{(}\DataTypeTok{x =} \KeywordTok{rep.int}\NormalTok{(}\DecValTok{1}\OperatorTok{:}\NormalTok{n.ticks,}\KeywordTok{max}\NormalTok{(L1)) , }\DataTypeTok{y =}\NormalTok{ value, }\DataTypeTok{group =}\NormalTok{ L1, }\DataTypeTok{colour=}\KeywordTok{factor}\NormalTok{(L1)), ) }\OperatorTok{+}
\StringTok{  }\KeywordTok{geom_line}\NormalTok{(}\DataTypeTok{data =}\NormalTok{ y_m, }\KeywordTok{aes}\NormalTok{(}\DataTypeTok{x =} \KeywordTok{rep.int}\NormalTok{(}\DecValTok{1}\OperatorTok{:}\NormalTok{n.ticks,}\KeywordTok{max}\NormalTok{(L1)) , }\DataTypeTok{y =}\NormalTok{ value, }\DataTypeTok{group =}\NormalTok{ L1, }\DataTypeTok{colour=}\KeywordTok{factor}\NormalTok{(L1)), ) }\OperatorTok{+}
\StringTok{  }\CommentTok{#linetype=y_m$L1) +}
\StringTok{  }\KeywordTok{theme_tufte}\NormalTok{() }
\CommentTok{# +  geom_text(x=,y=,label = max(value),check_overlap = TUE)}
\end{Highlighting}
\end{Shaded}

Insert math expression in legend title

\begin{Shaded}
\begin{Highlighting}[]
\KeywordTok{ggplot}\NormalTok{() }\OperatorTok{+}\StringTok{ }
\StringTok{  }\KeywordTok{labs}\NormalTok{(}\DataTypeTok{title =} \KeywordTok{bquote}\NormalTok{(}\StringTok{"Hello"} \OperatorTok{~}\StringTok{ }\NormalTok{r[xy] }\OperatorTok{~}\StringTok{ "and"} \OperatorTok{~}\StringTok{ }\NormalTok{B}\OperatorTok{^}\DecValTok{2}\NormalTok{))}
\end{Highlighting}
\end{Shaded}

Create double line break with expression in legend title (and labels)

\begin{Shaded}
\begin{Highlighting}[]
\KeywordTok{ggplot}\NormalTok{() }\OperatorTok{+}
\StringTok{  }\KeywordTok{scale_color_manual}\NormalTok{(}
    \KeywordTok{expression}\NormalTok{(}\KeywordTok{atop}\NormalTok{(}\StringTok{"text"}\NormalTok{, }
    \KeywordTok{atop}\NormalTok{(}\KeywordTok{textstyle}\NormalTok{(epsilon))))}
\NormalTok{    )}
\end{Highlighting}
\end{Shaded}

\subsubsection{Lists}\label{lists}

Find maximum value in entire list

\begin{Shaded}
\begin{Highlighting}[]
\NormalTok{master <-}\StringTok{ }\KeywordTok{list}\NormalTok{(}\DecValTok{1}\OperatorTok{:}\DecValTok{10}\NormalTok{,}\DecValTok{100}\NormalTok{,}\KeywordTok{rnorm}\NormalTok{(}\DecValTok{12}\NormalTok{))}
\KeywordTok{do.call}\NormalTok{(max,master)}
\end{Highlighting}
\end{Shaded}

Plot all elements in a list

\begin{Shaded}
\begin{Highlighting}[]
\NormalTok{xx <-}\StringTok{ }\KeywordTok{list}\NormalTok{(}\KeywordTok{sample}\NormalTok{(}\DecValTok{5}\NormalTok{,}\DecValTok{1000}\NormalTok{,}\DataTypeTok{replace=}\NormalTok{T),}\KeywordTok{rnorm}\NormalTok{(}\DecValTok{1000}\NormalTok{),}\KeywordTok{sample}\NormalTok{(}\DecValTok{50}\NormalTok{,}\DecValTok{1000}\NormalTok{,}\DataTypeTok{replace=}\NormalTok{T))}
\KeywordTok{plot}\NormalTok{(}\KeywordTok{unlist}\NormalTok{(xx),}\DataTypeTok{type=}\StringTok{"l"}\NormalTok{)}
\end{Highlighting}
\end{Shaded}

\subsubsection{Loops}\label{loops}

Save loop output in master list

\begin{Shaded}
\begin{Highlighting}[]
\NormalTok{pars <-}\StringTok{ }\KeywordTok{seq}\NormalTok{(}\DecValTok{0}\NormalTok{,}\DecValTok{1}\NormalTok{,}\FloatTok{0.5}\NormalTok{)}
\NormalTok{master <-}\StringTok{ }\KeywordTok{list}\NormalTok{()}
\NormalTok{t_list <-}\StringTok{ }\KeywordTok{list}\NormalTok{()}
\ControlFlowTok{for}\NormalTok{ (p }\ControlFlowTok{in} \DecValTok{1}\OperatorTok{:}\KeywordTok{length}\NormalTok{(pars))\{}
  \ControlFlowTok{for}\NormalTok{(t }\ControlFlowTok{in} \DecValTok{5}\NormalTok{)\{}
\NormalTok{    tt <-}\StringTok{ }\KeywordTok{rnorm}\NormalTok{(}\DecValTok{1000}\OperatorTok{*}\NormalTok{t)}
\NormalTok{    t_list[t] <-tt }
\NormalTok{  \}}
\NormalTok{  master[[}\KeywordTok{length}\NormalTok{(master)}\OperatorTok{+}\DecValTok{1}\NormalTok{]] <-}\StringTok{ }\NormalTok{t_list }\CommentTok{# store in master list}
\NormalTok{\} }
\end{Highlighting}
\end{Shaded}

\subsubsection{Messages}\label{messages}

Display status message of progress

\begin{Shaded}
\begin{Highlighting}[]
\ControlFlowTok{for}\NormalTok{(i }\ControlFlowTok{in} \DecValTok{1}\OperatorTok{:}\DecValTok{10}\NormalTok{) \{}
  \KeywordTok{Sys.sleep}\NormalTok{(}\FloatTok{0.2}\NormalTok{)}
  \CommentTok{# Dirk says using cat() like this is naughty ;-)}
  \CommentTok{#cat(i,"\textbackslash{}r")}
  \CommentTok{# So you can use message() like this, thanks to Sharpie's}
  \CommentTok{# comment to use appendLF=FALSE.}
  \KeywordTok{message}\NormalTok{(i,}\StringTok{"}\CharTok{\textbackslash{}r}\StringTok{"}\NormalTok{,}\DataTypeTok{appendLF=}\OtherTok{FALSE}\NormalTok{) }\CommentTok{# appendLF = new line }
  \KeywordTok{flush.console}\NormalTok{()}
\NormalTok{\}}
\end{Highlighting}
\end{Shaded}

Display popup progress bar

\begin{Shaded}
\begin{Highlighting}[]
\KeywordTok{require}\NormalTok{(tcltk)}
\NormalTok{pb <-}\StringTok{ }\KeywordTok{tkProgressBar}\NormalTok{(}\StringTok{"test progress bar"}\NormalTok{, }\StringTok{"Some information in %"}\NormalTok{,}
        \DecValTok{0}\NormalTok{, }\DecValTok{100}\NormalTok{, }\DecValTok{50}\NormalTok{)}
\KeywordTok{Sys.sleep}\NormalTok{(}\FloatTok{0.5}\NormalTok{)}
\NormalTok{u <-}\StringTok{ }\KeywordTok{c}\NormalTok{(}\DecValTok{0}\NormalTok{, }\KeywordTok{sort}\NormalTok{(}\KeywordTok{runif}\NormalTok{(}\DecValTok{20}\NormalTok{, }\DecValTok{0}\NormalTok{ ,}\DecValTok{100}\NormalTok{)), }\DecValTok{100}\NormalTok{)}
\ControlFlowTok{for}\NormalTok{(i }\ControlFlowTok{in}\NormalTok{ u) \{}
    \KeywordTok{Sys.sleep}\NormalTok{(}\FloatTok{0.1}\NormalTok{)}
\NormalTok{    info <-}\StringTok{ }\KeywordTok{sprintf}\NormalTok{(}\StringTok{"%d%% done"}\NormalTok{, }\KeywordTok{round}\NormalTok{(i))}
    \KeywordTok{setTkProgressBar}\NormalTok{(pb, i, }\KeywordTok{sprintf}\NormalTok{(}\StringTok{"test (%s)"}\NormalTok{, info), info)}
\NormalTok{\}}
\KeywordTok{Sys.sleep}\NormalTok{(}\DecValTok{5}\NormalTok{)}
\KeywordTok{close}\NormalTok{(pb)}
\end{Highlighting}
\end{Shaded}

\subsubsection{\texorpdfstring{\texttt{NAs}}{NAs}}\label{nas}

Replace \texttt{NAs} with 0's

\begin{Shaded}
\begin{Highlighting}[]
\NormalTok{df[}\KeywordTok{is.na}\NormalTok{(df)] <-}\StringTok{ }\DecValTok{0}
\end{Highlighting}
\end{Shaded}

Replace X values less than given value (V) with 0

\begin{Shaded}
\begin{Highlighting}[]
\NormalTok{df}\OperatorTok{$}\NormalTok{X[df}\OperatorTok{$}\NormalTok{X}\OperatorTok{<}\NormalTok{V] <-}\StringTok{ }\DecValTok{0} 
\end{Highlighting}
\end{Shaded}

Check for \texttt{NAs}

\begin{Shaded}
\begin{Highlighting}[]
\KeywordTok{sapply}\NormalTok{(df, }\ControlFlowTok{function}\NormalTok{(x) }\KeywordTok{sum}\NormalTok{(}\KeywordTok{is.na}\NormalTok{(x)))}
\end{Highlighting}
\end{Shaded}

Replace \texttt{NaN} and \texttt{Inf} values with \texttt{NA}

\begin{Shaded}
\begin{Highlighting}[]
\NormalTok{df}\OperatorTok{$}\NormalTok{col1[}\KeywordTok{which}\NormalTok{(}\OperatorTok{!}\KeywordTok{is.finite}\NormalTok{(df}\OperatorTok{$}\NormalTok{col1))] <-}\StringTok{  }\OtherTok{NA}
\end{Highlighting}
\end{Shaded}

Fill in missing data values in sequence with \texttt{NA}

\begin{Shaded}
\begin{Highlighting}[]
\CommentTok{# /Users/malishev/Documents/Manuscripts/Chapter4/Sims/Chapter4_figs.R}
\KeywordTok{library}\NormalTok{(zoo)}
\NormalTok{data <-}\StringTok{ }\KeywordTok{data.frame}\NormalTok{(}\DataTypeTok{index =} \KeywordTok{c}\NormalTok{(}\DecValTok{1}\OperatorTok{:}\DecValTok{4}\NormalTok{, }\DecValTok{6}\OperatorTok{:}\DecValTok{10}\NormalTok{),}
  \DataTypeTok{data =} \KeywordTok{c}\NormalTok{(}\FloatTok{1.5}\NormalTok{,}\FloatTok{4.3}\NormalTok{,}\FloatTok{5.6}\NormalTok{,}\FloatTok{6.7}\NormalTok{,}\FloatTok{7.1}\NormalTok{,}\FloatTok{12.5}\NormalTok{,}\FloatTok{14.5}\NormalTok{,}\FloatTok{16.8}\NormalTok{,}\FloatTok{3.4}\NormalTok{))}
\CommentTok{#you can create a series}
\NormalTok{z <-}\StringTok{ }\KeywordTok{zoo}\NormalTok{(data}\OperatorTok{$}\NormalTok{data, data}\OperatorTok{$}\NormalTok{index)}
\CommentTok{#end extend it to the grid 1:10}
\NormalTok{z <-}\StringTok{ }\KeywordTok{merge}\NormalTok{(}\KeywordTok{zoo}\NormalTok{(,}\DecValTok{1}\OperatorTok{:}\DecValTok{10}\NormalTok{), z)}

\CommentTok{#worked example}
\CommentTok{# fill in missing Tb values }
\NormalTok{minTb.d <-}\StringTok{ }\KeywordTok{zoo}\NormalTok{(minTb}\OperatorTok{$}\NormalTok{Tick,minTb}\OperatorTok{$}\NormalTok{Days)}
\NormalTok{minTb.d <-}\StringTok{ }\KeywordTok{merge}\NormalTok{(}\KeywordTok{zoo}\NormalTok{(}\OtherTok{NULL}\NormalTok{,}\DecValTok{1}\OperatorTok{:}\NormalTok{days), minTb.d) }\CommentTok{# make the minTb series match the temp series (117 days)}
\NormalTok{minTb.d <-}\StringTok{ }\KeywordTok{as.numeric}\NormalTok{(minTb.d) }\CommentTok{# = time individuals reached VTMIN in ticks}
\NormalTok{minTb <-}\StringTok{ }\NormalTok{minTb.d }\OperatorTok{-}\StringTok{ }\NormalTok{temp}\OperatorTok{$}\NormalTok{Tick }\CommentTok{# get diff between starting time and time to reach VTMIN}
\NormalTok{minTb <-}\StringTok{ }\NormalTok{minTb}\OperatorTok{/}\DecValTok{2} \CommentTok{# convert ticks to minutes}
\NormalTok{minTb <-}\StringTok{ }\NormalTok{minTb}\OperatorTok{/}\DecValTok{60} \CommentTok{#convert to hours}
\NormalTok{minTb <-}\StringTok{ }\KeywordTok{data.frame}\NormalTok{(}\StringTok{"Days"}\NormalTok{=}\DecValTok{1}\OperatorTok{:}\NormalTok{days,}\StringTok{"Time"}\NormalTok{=minTb)}

\CommentTok{# then fill in missing values}
\KeywordTok{approx}\NormalTok{(minTb}\OperatorTok{$}\NormalTok{Time,}\DataTypeTok{method =} \StringTok{"linear"}\NormalTok{)}
\end{Highlighting}
\end{Shaded}

Remove rows with NA

\begin{Shaded}
\begin{Highlighting}[]
\NormalTok{data <-}\StringTok{ }\NormalTok{data[}\OperatorTok{!}\KeywordTok{is.na}\NormalTok{(data}\OperatorTok{$}\NormalTok{X),]}
\end{Highlighting}
\end{Shaded}

Turn NULLs in list into NAs to get numeric values (fix for `cannot
coerce double' error)

\begin{Shaded}
\begin{Highlighting}[]
\NormalTok{hl_list <-}\StringTok{ }\KeywordTok{lapply}\NormalTok{(hl_list, }\ControlFlowTok{function}\NormalTok{(x) }\KeywordTok{ifelse}\NormalTok{(x }\OperatorTok{==}\StringTok{ "NULL"}\NormalTok{, }\OtherTok{NA}\NormalTok{, x))}
\end{Highlighting}
\end{Shaded}

Turn NaN or NAs in list into 0s

\begin{Shaded}
\begin{Highlighting}[]
\CommentTok{# NaN}
\NormalTok{global_output <-}\StringTok{ }\KeywordTok{rapply}\NormalTok{(global_output, }\DataTypeTok{f=}\ControlFlowTok{function}\NormalTok{(x) }\KeywordTok{ifelse}\NormalTok{(}\KeywordTok{is.nan}\NormalTok{(x),}\DecValTok{0}\NormalTok{,x), }\DataTypeTok{how=}\StringTok{"replace"}\NormalTok{ )}

\CommentTok{# NA}
\NormalTok{global_output <-}\StringTok{ }\KeywordTok{rapply}\NormalTok{(global_output, }\DataTypeTok{f=}\ControlFlowTok{function}\NormalTok{(x) }\KeywordTok{ifelse}\NormalTok{(}\KeywordTok{is.na}\NormalTok{(x),}\DecValTok{0}\NormalTok{,x), }\DataTypeTok{how=}\StringTok{"replace"}\NormalTok{ )}
\end{Highlighting}
\end{Shaded}

\subsubsection{Packages}\label{packages}

\href{https://blog.socialcops.com/technology/data-science/announcing-rlandsat-landsat-8-data/}{\texttt{rLandsat}}\\
Sourcing, requesting, and downloading NASA Landsat 8 satellite data.

\href{https://blog.rstudio.com/2018/09/19/radix-for-r-markdown/}{Radix}\\
Improved \texttt{RMarkdown} output and interaction.

\href{http://www.stats.gla.ac.uk/~adrian/rpanel/}{\texttt{rpanel}}\\
\href{https://cran.r-project.org/doc/Rnews/Rnews_2006-4.pdf}{Reference
guide}\\
Create interactive GUI control toggles from \texttt{R}. Like an early
Shiny.

\subsubsection{Plotting}\label{plotting}

Plot one plot window above and two below

\begin{Shaded}
\begin{Highlighting}[]
\KeywordTok{layout}\NormalTok{(}\KeywordTok{matrix}\NormalTok{(}\KeywordTok{c}\NormalTok{(}\DecValTok{1}\NormalTok{,}\DecValTok{1}\NormalTok{,}\DecValTok{2}\NormalTok{,}\DecValTok{3}\NormalTok{), }\DecValTok{2}\NormalTok{, }\DecValTok{2}\NormalTok{, }\DataTypeTok{byrow =} \OtherTok{TRUE}\NormalTok{))}
\end{Highlighting}
\end{Shaded}

Bookend axis ticks for plot E.g. at 0 and 100 when data is 1:99

\begin{Shaded}
\begin{Highlighting}[]
\KeywordTok{axis}\NormalTok{(}\DecValTok{1}\NormalTok{,}\DataTypeTok{at=}\KeywordTok{c}\NormalTok{(}\DecValTok{0}\NormalTok{,}\KeywordTok{length}\NormalTok{(loco}\OperatorTok{$}\NormalTok{X)),}\DataTypeTok{labels=}\KeywordTok{c}\NormalTok{(}\StringTok{""}\NormalTok{,}\StringTok{""}\NormalTok{))}\CommentTok{# bookending axis tick marks}
\end{Highlighting}
\end{Shaded}

Optimal legend formatting for base

\begin{Shaded}
\begin{Highlighting}[]
\KeywordTok{legend}\NormalTok{(}\StringTok{"right"}\NormalTok{,}\DataTypeTok{legend=}\KeywordTok{c}\NormalTok{(}\StringTok{"Small"}\NormalTok{,}\StringTok{"Intermediate"}\NormalTok{,}\StringTok{"Large"}\NormalTok{),}\DataTypeTok{col=}\KeywordTok{c}\NormalTok{(colfunc[colvec[}\DecValTok{1}\OperatorTok{:}\DecValTok{3}\NormalTok{]]),}
       \DataTypeTok{bty=}\StringTok{"n"}\NormalTok{,}\DataTypeTok{pch=}\DecValTok{20}\NormalTok{,}\DataTypeTok{pt.cex=}\FloatTok{1.5}\NormalTok{,}\DataTypeTok{cex=}\FloatTok{0.7}\NormalTok{,}\DataTypeTok{y.intersp =} \FloatTok{0.5}\NormalTok{, }\DataTypeTok{xjust =} \FloatTok{0.5}\NormalTok{,}
       \DataTypeTok{title=}\StringTok{"Size class"}\NormalTok{,}\DataTypeTok{title.adj =} \FloatTok{0.3}\NormalTok{,}\DataTypeTok{text.font=}\DecValTok{2}\NormalTok{,}
       \DataTypeTok{trace=}\NormalTok{T,}\DataTypeTok{inset=}\FloatTok{0.1}\NormalTok{)}
\end{Highlighting}
\end{Shaded}

Plot inset plot in current plot
(\url{https://stackoverflow.com/questions/17041246/how-to-add-an-inset-subplot-to-topright-of-an-r-plot})

\begin{Shaded}
\begin{Highlighting}[]
\CommentTok{# calculate position of inset}
\NormalTok{plotdim <-}\StringTok{ }\KeywordTok{par}\NormalTok{(}\StringTok{"plt"}\NormalTok{)}\CommentTok{# get plot window dims as fraction of current plot dims }
\NormalTok{xleft    =}\StringTok{ }\NormalTok{plotdim[}\DecValTok{2}\NormalTok{] }\OperatorTok{-}\StringTok{ }\NormalTok{(plotdim[}\DecValTok{2}\NormalTok{] }\OperatorTok{-}\StringTok{ }\NormalTok{plotdim[}\DecValTok{1}\NormalTok{]) }\OperatorTok{*}\StringTok{ }\FloatTok{0.5}
\NormalTok{xright   =}\StringTok{ }\NormalTok{plotdim[}\DecValTok{2}\NormalTok{]  }\CommentTok{#}
\NormalTok{ybottom  =}\StringTok{ }\NormalTok{plotdim[}\DecValTok{4}\NormalTok{] }\OperatorTok{-}\StringTok{ }\NormalTok{(plotdim[}\DecValTok{4}\NormalTok{] }\OperatorTok{-}\StringTok{ }\NormalTok{plotdim[}\DecValTok{3}\NormalTok{]) }\OperatorTok{*}\StringTok{ }\FloatTok{0.5}  \CommentTok{#}
\NormalTok{ytop     =}\StringTok{ }\NormalTok{plotdim[}\DecValTok{4}\NormalTok{]  }\CommentTok{#}

\CommentTok{# set position for plot inset}
\KeywordTok{par}\NormalTok{(}\DataTypeTok{fig =} \KeywordTok{c}\NormalTok{(xleft, xright, ybottom, ytop),}\DataTypeTok{mar=}\KeywordTok{c}\NormalTok{(}\DecValTok{0}\NormalTok{,}\DecValTok{0}\NormalTok{,}\DecValTok{0}\NormalTok{,}\DecValTok{0}\NormalTok{),}\DataTypeTok{new=}\OtherTok{TRUE}\NormalTok{)}

\KeywordTok{boxplot}\NormalTok{(Eggs}\OperatorTok{~}\NormalTok{Size,}\DataTypeTok{data=}\NormalTok{meso2,}
                \DataTypeTok{col=}\KeywordTok{adjustcolor}\NormalTok{(colfunc[colvec[}\DecValTok{1}\OperatorTok{:}\DecValTok{3}\NormalTok{]],}\DataTypeTok{alpha=}\FloatTok{0.5}\NormalTok{),}
                \DataTypeTok{notch =}\NormalTok{ T,}\DataTypeTok{xlab=}\StringTok{"Week"}\NormalTok{,}\DataTypeTok{ylab=}\StringTok{"Diameter (mm)"}\NormalTok{,}
                \DataTypeTok{xaxs =} \StringTok{"i"}\NormalTok{, }\DataTypeTok{yaxs =} \StringTok{"i"}
\NormalTok{                ) }
\end{Highlighting}
\end{Shaded}

Interactive plots with rCharts (javascript and d3 viz)\\
\url{http://ramnathv.github.io/rCharts/}

\begin{Shaded}
\begin{Highlighting}[]
\KeywordTok{require}\NormalTok{(devtools)}
\KeywordTok{install_github}\NormalTok{(}\StringTok{'rCharts'}\NormalTok{, }\StringTok{'ramnathv'}\NormalTok{)}
\end{Highlighting}
\end{Shaded}

Cluster plot\\
\url{https://rpubs.com/dgrtwo/technology-clusters}

\begin{Shaded}
\begin{Highlighting}[]
\KeywordTok{library}\NormalTok{(readr)}
\KeywordTok{library}\NormalTok{(dplyr)}
\KeywordTok{library}\NormalTok{(igraph)}
\KeywordTok{library}\NormalTok{(ggraph)}
\KeywordTok{library}\NormalTok{(ggforce)}

\CommentTok{# This shared file contains the number of question that have each pair of tags}
\CommentTok{# This counts only questions that are not deleted and have a positive score}
\NormalTok{tag_pair_data <-}\StringTok{ }\KeywordTok{read_csv}\NormalTok{(}\StringTok{"http://varianceexplained.org/files/tag_pairs.csv.gz"}\NormalTok{)}

\NormalTok{relationships <-}\StringTok{ }\NormalTok{tag_pair_data }\OperatorTok
\StringTok{  }\KeywordTok{mutate}\NormalTok{(}\DataTypeTok{Fraction =}\NormalTok{ Cooccur }\OperatorTok{/}\StringTok{ }\NormalTok{Tag1Total) }\OperatorTok
\StringTok{  }\KeywordTok{filter}\NormalTok{(Fraction }\OperatorTok{>=}\StringTok{ }\NormalTok{.}\DecValTok{35}\NormalTok{) }\OperatorTok
\StringTok{  }\KeywordTok{distinct}\NormalTok{(Tag1)}

\NormalTok{v <-}\StringTok{ }\NormalTok{tag_pair_data }\OperatorTok
\StringTok{  }\KeywordTok{select}\NormalTok{(Tag1, Tag1Total) }\OperatorTok
\StringTok{  }\KeywordTok{distinct}\NormalTok{(Tag1) }\OperatorTok
\StringTok{  }\KeywordTok{filter}\NormalTok{(Tag1 }\OperatorTok\StringTok{ }\NormalTok{relationships}\OperatorTok{$}\NormalTok{Tag1 }\OperatorTok{|}
\StringTok{         }\NormalTok{Tag1 }\OperatorTok\StringTok{ }\NormalTok{relationships}\OperatorTok{$}\NormalTok{Tag2) }\OperatorTok
\StringTok{  }\KeywordTok{arrange}\NormalTok{(}\KeywordTok{desc}\NormalTok{(Tag1Total))}

\NormalTok{a <-}\StringTok{ }\NormalTok{grid}\OperatorTok{::}\KeywordTok{arrow}\NormalTok{(}\DataTypeTok{length =}\NormalTok{ grid}\OperatorTok{::}\KeywordTok{unit}\NormalTok{(.}\DecValTok{08}\NormalTok{, }\StringTok{"inches"}\NormalTok{), }\DataTypeTok{ends =} \StringTok{"first"}\NormalTok{, }\DataTypeTok{type =} \StringTok{"closed"}\NormalTok{)}

\KeywordTok{set.seed}\NormalTok{(}\DecValTok{2016}\NormalTok{)}

\NormalTok{relationships }\OperatorTok
\StringTok{  }\KeywordTok{graph_from_data_frame}\NormalTok{(}\DataTypeTok{vertices =}\NormalTok{ v) }\OperatorTok
\StringTok{  }\KeywordTok{ggraph}\NormalTok{(}\DataTypeTok{layout =} \StringTok{"fr"}\NormalTok{) }\OperatorTok{+}
\StringTok{  }\KeywordTok{geom_edge_link}\NormalTok{(}\KeywordTok{aes}\NormalTok{(}\DataTypeTok{alpha =}\NormalTok{ Fraction), }\DataTypeTok{arrow =}\NormalTok{ a) }\OperatorTok{+}
\StringTok{  }\KeywordTok{geom_node_point}\NormalTok{(}\KeywordTok{aes}\NormalTok{(}\DataTypeTok{size =}\NormalTok{ Tag1Total), }\DataTypeTok{color =} \StringTok{"lightblue"}\NormalTok{) }\OperatorTok{+}
\StringTok{  }\KeywordTok{geom_node_text}\NormalTok{(}\KeywordTok{aes}\NormalTok{(}\DataTypeTok{size =}\NormalTok{ Tag1Total, }\DataTypeTok{label =}\NormalTok{ name), }\DataTypeTok{check_overlap =} \OtherTok{TRUE}\NormalTok{) }\OperatorTok{+}
\StringTok{  }\KeywordTok{scale_size_continuous}\NormalTok{(}\DataTypeTok{range =} \KeywordTok{c}\NormalTok{(}\DecValTok{2}\NormalTok{, }\DecValTok{9}\NormalTok{)) }\OperatorTok{+}
\StringTok{  }\NormalTok{ggforce}\OperatorTok{::}\KeywordTok{theme_no_axes}\NormalTok{() }\OperatorTok{+}
\StringTok{  }\KeywordTok{theme}\NormalTok{(}\DataTypeTok{legend.position =} \StringTok{"none"}\NormalTok{)}
\end{Highlighting}
\end{Shaded}

Define global plotting graphics function.

The \texttt{plot\_it.R} function is updated on the
\href{https://raw.githubusercontent.com/darwinanddavis/plot_it/master/plot_it.R}{plot\_it
Github page}.

\begin{Shaded}
\begin{Highlighting}[]
\KeywordTok{require}\NormalTok{(ggplot2)}
\KeywordTok{require}\NormalTok{(ggthemes)}
\NormalTok{### set plotting params   }
\NormalTok{plot_it <-}\StringTok{ }\ControlFlowTok{function}\NormalTok{(manuscript,bg,cp1,cp2,alpha,family)\{ }\CommentTok{# plotting function (plot for MS or not, set bg color, set color palette from RColorBrewer, set alpha value for transperancy) }
  \KeywordTok{graphics.off}\NormalTok{()}
  \ControlFlowTok{if}\NormalTok{(manuscript}\OperatorTok{==}\DecValTok{0}\NormalTok{)\{}
    \ControlFlowTok{if}\NormalTok{(bg}\OperatorTok{==}\StringTok{"black"}\NormalTok{)\{}
\NormalTok{      colvec <<-}\StringTok{ }\KeywordTok{magma}\NormalTok{(}\DecValTok{200}\NormalTok{,}\DecValTok{1}\NormalTok{) }\CommentTok{# plot window bg # USES <<- OPERATOR}
      \KeywordTok{par}\NormalTok{(}\DataTypeTok{bg =}\NormalTok{ colvec[}\DecValTok{1}\NormalTok{],}\DataTypeTok{col.axis=}\StringTok{"white"}\NormalTok{,}\DataTypeTok{col.lab=}\StringTok{"white"}\NormalTok{,}\DataTypeTok{col.main=}\StringTok{"white"}\NormalTok{,}\DataTypeTok{fg=}\StringTok{"white"}\NormalTok{,}\DataTypeTok{bty=}\StringTok{"n"}\NormalTok{,}\DataTypeTok{las=}\DecValTok{1}\NormalTok{,}\DataTypeTok{mar=}\KeywordTok{c}\NormalTok{(}\DecValTok{5}\NormalTok{,}\DecValTok{6}\NormalTok{,}\DecValTok{4}\NormalTok{,}\DecValTok{2}\NormalTok{),}\DataTypeTok{family=}\NormalTok{family) }\CommentTok{#mono}
\NormalTok{      border=}\KeywordTok{adjustcolor}\NormalTok{(}\StringTok{"purple"}\NormalTok{,}\DataTypeTok{alpha=}\FloatTok{0.5}\NormalTok{)}
\NormalTok{    \}}\ControlFlowTok{else}\NormalTok{\{}
\NormalTok{      colvec <<-}\StringTok{ }\KeywordTok{bpy.colors}\NormalTok{(}\DecValTok{200}\NormalTok{) }\CommentTok{# plot window bg # USES <<- OPERATOR}
      \KeywordTok{par}\NormalTok{(}\DataTypeTok{bg =}\NormalTok{ colvec[}\DecValTok{1}\NormalTok{],}\DataTypeTok{col.axis=}\StringTok{"white"}\NormalTok{,}\DataTypeTok{col.lab=}\StringTok{"white"}\NormalTok{,}\DataTypeTok{col.main=}\StringTok{"white"}\NormalTok{,}\DataTypeTok{fg=}\StringTok{"white"}\NormalTok{,}\DataTypeTok{bty=}\StringTok{"n"}\NormalTok{,}\DataTypeTok{las=}\DecValTok{1}\NormalTok{,}\DataTypeTok{mar=}\KeywordTok{c}\NormalTok{(}\DecValTok{5}\NormalTok{,}\DecValTok{6}\NormalTok{,}\DecValTok{4}\NormalTok{,}\DecValTok{2}\NormalTok{),}\DataTypeTok{family=}\NormalTok{family) }
\NormalTok{      border=}\KeywordTok{adjustcolor}\NormalTok{(}\StringTok{"blue"}\NormalTok{,}\DataTypeTok{alpha=}\FloatTok{0.5}\NormalTok{)}
\NormalTok{    \}}
\NormalTok{  \}}\ControlFlowTok{else}\NormalTok{\{}
    \CommentTok{#    graphics.off()}
    \KeywordTok{par}\NormalTok{(}\DataTypeTok{bty=}\StringTok{"n"}\NormalTok{,}\DataTypeTok{las=}\DecValTok{1}\NormalTok{,}\DataTypeTok{family=}\NormalTok{family) }
\NormalTok{    colv<-}\StringTok{"white"}
\NormalTok{  \}}
  \CommentTok{# color palettes}
  \CommentTok{# ifelse(manuscript==1,colvec<-adjustcolor(brewer.pal(9,cp1)[9], alpha = alpha),colvec <- adjustcolor(brewer.pal(9,cp1)[5], alpha = alpha)) # fine tune plotting colors for plotting bg}
  \CommentTok{# colfunc <<- colorRampPalette(brewer.pal(9,cp1),alpha=alpha)}
\NormalTok{  cp1_info <-}\StringTok{ }\NormalTok{brewer.pal.info[cp1,]}\OperatorTok{$}\NormalTok{maxcolors}
\NormalTok{  cp2_info <-}\StringTok{ }\NormalTok{brewer.pal.info[cp2,]}\OperatorTok{$}\NormalTok{maxcolors}
\NormalTok{  colv <<-}\StringTok{ }\KeywordTok{brewer.pal}\NormalTok{(cp1_info,cp1) }\CommentTok{# USES <<- OPERATOR}
\NormalTok{  colv2 <<-}\StringTok{ }\KeywordTok{brewer.pal}\NormalTok{(cp2_info,cp2) }\CommentTok{# USES <<- OPERATOR}
\NormalTok{\}}

\CommentTok{# Setting ggplot theme graphics}
\NormalTok{plot_it_gg <-}\StringTok{ }\ControlFlowTok{function}\NormalTok{(bg)\{ }\CommentTok{# bg = colour to plot bg, family = font family}
  \ControlFlowTok{if}\NormalTok{(bg}\OperatorTok{==}\StringTok{"white"}\NormalTok{)\{}
\NormalTok{    bg <-}\StringTok{ "white"}
\NormalTok{    fg <-}\StringTok{ "black"}
    \KeywordTok{theme_tufte}\NormalTok{(}\DataTypeTok{base_family =} \StringTok{"HersheySans"}\NormalTok{) }\OperatorTok{+}
\StringTok{      }\KeywordTok{theme}\NormalTok{(}\DataTypeTok{panel.border =} \KeywordTok{element_blank}\NormalTok{(),}\DataTypeTok{panel.grid.major =} \KeywordTok{element_blank}\NormalTok{(),}\DataTypeTok{panel.grid.minor =} \KeywordTok{element_blank}\NormalTok{(),}\DataTypeTok{panel.background =} \KeywordTok{element_rect}\NormalTok{(}\DataTypeTok{fill =}\NormalTok{ bg,}\DataTypeTok{colour =}\NormalTok{ bg),}\DataTypeTok{plot.background =} \KeywordTok{element_rect}\NormalTok{(}\DataTypeTok{fill=}\NormalTok{bg)) }\OperatorTok{+}
\StringTok{      }\KeywordTok{theme}\NormalTok{(}\DataTypeTok{axis.line =} \KeywordTok{element_line}\NormalTok{(}\DataTypeTok{color =}\NormalTok{ fg)) }\OperatorTok{+}\KeywordTok{theme}\NormalTok{(}\DataTypeTok{axis.ticks =} \KeywordTok{element_line}\NormalTok{(}\DataTypeTok{color =}\NormalTok{ fg)) }\OperatorTok{+}\StringTok{ }\KeywordTok{theme}\NormalTok{(}\DataTypeTok{plot.title =} \KeywordTok{element_text}\NormalTok{(}\DataTypeTok{colour =}\NormalTok{ fg)) }\OperatorTok{+}\KeywordTok{theme}\NormalTok{(}\DataTypeTok{axis.title.x =} \KeywordTok{element_text}\NormalTok{(}\DataTypeTok{colour =}\NormalTok{ fg), }\DataTypeTok{axis.title.y =} \KeywordTok{element_text}\NormalTok{(}\DataTypeTok{colour =}\NormalTok{ fg)) }\OperatorTok{+}\StringTok{ }\KeywordTok{theme}\NormalTok{(}\DataTypeTok{axis.text.x =} \KeywordTok{element_text}\NormalTok{(}\DataTypeTok{color =}\NormalTok{ fg),}\DataTypeTok{axis.text.y =} \KeywordTok{element_text}\NormalTok{(}\DataTypeTok{color =}\NormalTok{ fg)) }\OperatorTok{+}\StringTok{ }\KeywordTok{theme}\NormalTok{(}\DataTypeTok{legend.key =} \KeywordTok{element_rect}\NormalTok{(}\DataTypeTok{fill =}\NormalTok{ bg)) }\OperatorTok{+}\StringTok{ }\KeywordTok{theme}\NormalTok{(}\DataTypeTok{legend.title =} \KeywordTok{element_text}\NormalTok{(}\DataTypeTok{colour=}\NormalTok{fg)) }\OperatorTok{+}\StringTok{ }\KeywordTok{theme}\NormalTok{(}\DataTypeTok{legend.text =} \KeywordTok{element_text}\NormalTok{(}\DataTypeTok{colour=}\NormalTok{fg))}
\NormalTok{  \}}
\NormalTok{\}}\CommentTok{# end gg}

\NormalTok{### Set plotting function  }
\end{Highlighting}
\end{Shaded}

\begin{Shaded}
\begin{Highlighting}[]
\KeywordTok{require}\NormalTok{(}\StringTok{"RCurl"}\NormalTok{)}
\NormalTok{script <-}\StringTok{ }\KeywordTok{getURL}\NormalTok{(}\StringTok{"https://raw.githubusercontent.com/darwinanddavis/plot_it/master/plot_it.R"}\NormalTok{, }\DataTypeTok{ssl.verifypeer =} \OtherTok{FALSE}\NormalTok{)}
\KeywordTok{eval}\NormalTok{(}\KeywordTok{parse}\NormalTok{(}\DataTypeTok{text =}\NormalTok{ script))}

\KeywordTok{cat}\NormalTok{(}\StringTok{"plot_it( }\CharTok{\textbackslash{}n}\StringTok{0 for presentation, 1 for manuscript, }\CharTok{\textbackslash{}n}\StringTok{set colour for background, }\CharTok{\textbackslash{}n}\StringTok{set colour palette. use 'display.brewer.all()', }\CharTok{\textbackslash{}n}\StringTok{set alpha for colour transperancy, }\CharTok{\textbackslash{}n}\StringTok{set font style }\CharTok{\textbackslash{}n}\StringTok{)"}\NormalTok{)}
\KeywordTok{plot_it}\NormalTok{(}\DecValTok{0}\NormalTok{,}\StringTok{"blue"}\NormalTok{,}\StringTok{"Spectral"}\NormalTok{,}\StringTok{"Greens"}\NormalTok{,}\DecValTok{1}\NormalTok{,}\StringTok{"mono"}\NormalTok{) }\CommentTok{# set col function params}
\KeywordTok{plot_it_gg}\NormalTok{(}\StringTok{"white"}\NormalTok{) }\CommentTok{# same as above }
\end{Highlighting}
\end{Shaded}

Make plot cycle on one page

\begin{Shaded}
\begin{Highlighting}[]
\KeywordTok{plot}\NormalTok{(m_abundance}\OperatorTok{$}\NormalTok{gam,}\DataTypeTok{pages=}\DecValTok{1}\NormalTok{)}
\end{Highlighting}
\end{Shaded}

Get plot summaries and values from plot

\begin{Shaded}
\begin{Highlighting}[]
\KeywordTok{plot.gam}\NormalTok{(m_abundance}\OperatorTok{$}\NormalTok{gam,}\DataTypeTok{shade=}\NormalTok{T,}\DataTypeTok{pages=}\DecValTok{1}\NormalTok{,}\DataTypeTok{seWithMean =}\NormalTok{ T)[}\DecValTok{1}\NormalTok{] }\CommentTok{# everything}
\KeywordTok{plot.gam}\NormalTok{(m_abundance}\OperatorTok{$}\NormalTok{gam,}\DataTypeTok{shade=}\NormalTok{T,}\DataTypeTok{pages=}\DecValTok{1}\NormalTok{,}\DataTypeTok{seWithMean =}\NormalTok{ T)[}\DecValTok{1}\NormalTok{][[}\DecValTok{1}\NormalTok{]]}\OperatorTok{$}\NormalTok{x }\CommentTok{#subset x}
\KeywordTok{plot.gam}\NormalTok{(m_abundance}\OperatorTok{$}\NormalTok{gam,}\DataTypeTok{shade=}\NormalTok{T,}\DataTypeTok{pages=}\DecValTok{1}\NormalTok{,}\DataTypeTok{seWithMean =}\NormalTok{ T)[}\DecValTok{1}\NormalTok{][[}\DecValTok{1}\NormalTok{]]}\OperatorTok{$}\NormalTok{fit }\CommentTok{#get values to produce fit curve }
\end{Highlighting}
\end{Shaded}

Package for stock world maps

\begin{Shaded}
\begin{Highlighting}[]
\CommentTok{#worldmap}
\KeywordTok{library}\NormalTok{(choroplethrMaps)}
\end{Highlighting}
\end{Shaded}

Circle packing, tree, dendogram, network plots

\begin{Shaded}
\begin{Highlighting}[]
\CommentTok{# dendogram tree nested bubble circle packing network }
\CommentTok{# https://www.r-graph-gallery.com/313-basic-circle-packing-with-several-levels/}

\CommentTok{# circle packing plot  }
\CommentTok{# Libraries}
\NormalTok{p <-}\StringTok{ }\KeywordTok{c}\NormalTok{(}\StringTok{"ggraph"}\NormalTok{,}\StringTok{"igraph"}\NormalTok{,}\StringTok{"tidyverse"}\NormalTok{,}\StringTok{"DeducerSpatial"}\NormalTok{,}\StringTok{"Rcpp"}\NormalTok{,}\StringTok{"car"}\NormalTok{)}
\KeywordTok{install.packages}\NormalTok{(p,}\DataTypeTok{dependencies =}\NormalTok{ T)}
\KeywordTok{lapply}\NormalTok{(p,library,}\DataTypeTok{character.only=}\NormalTok{T)}

\CommentTok{# We need a data frame giving a hierarchical structure. Let's consider the flare dataset:}
\NormalTok{edges=flare}\OperatorTok{$}\NormalTok{edges}
\CommentTok{# edges cols = character}

\CommentTok{# Usually we associate another dataset that give information about each node of the dataset:}
\NormalTok{vertices =}\StringTok{ }\NormalTok{flare}\OperatorTok{$}\NormalTok{vertices}
\CommentTok{# vertices cols = character, numeric, character}

\CommentTok{# Create a subset of the dataset (I remove 1 level)}
\NormalTok{edges =}\StringTok{ }\NormalTok{flare}\OperatorTok{$}\NormalTok{edges }\OperatorTok\StringTok{ }\KeywordTok{filter}\NormalTok{(to }\OperatorTok\StringTok{ }\NormalTok{from) }\OperatorTok\StringTok{ }\KeywordTok{droplevels}\NormalTok{()}
\NormalTok{vertices =}\StringTok{ }\NormalTok{flare}\OperatorTok{$}\NormalTok{vertices }\OperatorTok\StringTok{ }\KeywordTok{filter}\NormalTok{(name }\OperatorTok\StringTok{ }\KeywordTok{c}\NormalTok{(edges}\OperatorTok{$}\NormalTok{from, edges}\OperatorTok{$}\NormalTok{to)) }\OperatorTok\StringTok{ }\KeywordTok{droplevels}\NormalTok{()}
\NormalTok{vertices}\OperatorTok{$}\NormalTok{size=}\KeywordTok{runif}\NormalTok{(}\KeywordTok{nrow}\NormalTok{(vertices))}

\CommentTok{# Then we have to make a 'graph' object using the igraph library:}
\NormalTok{mygraph <-}\StringTok{ }\KeywordTok{graph_from_data_frame}\NormalTok{( edges, }\DataTypeTok{vertices=}\NormalTok{vertices )}

\CommentTok{# circle packing}
\KeywordTok{ggraph}\NormalTok{(mygraph, }\DataTypeTok{layout =} \StringTok{'circlepack'}\NormalTok{, }\DataTypeTok{weight=}\StringTok{"size"}\NormalTok{,}\DataTypeTok{sort.by=}\OtherTok{NULL}\NormalTok{,}\DataTypeTok{direction=}\StringTok{"out"}\NormalTok{) }\OperatorTok{+}\StringTok{ }
\StringTok{  }\KeywordTok{geom_node_circle}\NormalTok{(}\KeywordTok{aes}\NormalTok{(}\DataTypeTok{fill=}\NormalTok{depth)) }\OperatorTok{+}
\StringTok{  }\KeywordTok{geom_node_text}\NormalTok{(}\KeywordTok{aes}\NormalTok{(}\DataTypeTok{label=}\NormalTok{shortName, }\DataTypeTok{filter=}\NormalTok{leaf, }\DataTypeTok{fill=}\NormalTok{depth, }\DataTypeTok{size=}\NormalTok{size)) }\OperatorTok{+}\StringTok{ }\CommentTok{# add text}
\StringTok{  }\CommentTok{# geom_node_label(aes(label=shortName, filter=leaf, size=size)) + # add text boxes}
\StringTok{  }\KeywordTok{theme_void}\NormalTok{() }\OperatorTok{+}\StringTok{ }
\StringTok{  }\CommentTok{# theme(legend.position="F") + #show legend}
\StringTok{  }\KeywordTok{scale_fill_viridis}\NormalTok{(}\DataTypeTok{alpha=}\FloatTok{0.5}\NormalTok{,}\DataTypeTok{direction=}\OperatorTok{-}\DecValTok{1}\NormalTok{,}\DataTypeTok{option=}\StringTok{"magma"}\NormalTok{) }\OperatorTok{+}
\StringTok{  }\CommentTok{# scale_fill_distiller(palette = "Blues")}

\CommentTok{#circular dendo}
\KeywordTok{str}\NormalTok{(mygraph)}
\KeywordTok{ggraph}\NormalTok{(mygraph, }\DataTypeTok{layout=}\StringTok{'dendrogram'}\NormalTok{, }\DataTypeTok{circular=}\NormalTok{T) }\OperatorTok{+}\StringTok{ }
\StringTok{  }\KeywordTok{geom_edge_diagonal}\NormalTok{(}\DataTypeTok{flipped=}\NormalTok{F,}
                     \DataTypeTok{label_colour =} \StringTok{"black"}\NormalTok{,}
                     \DataTypeTok{label_alpha =} \DecValTok{1}\NormalTok{,}
                     \DataTypeTok{angle_calc =} \StringTok{"rot"}\NormalTok{,}
                     \DataTypeTok{force_flip =} \OtherTok{TRUE}\NormalTok{,}\DataTypeTok{label_dodge =} \OtherTok{NULL}\NormalTok{, }\DataTypeTok{label_push =} \OtherTok{NULL}\NormalTok{, }
                     \DataTypeTok{show.legend =} \OtherTok{NA}\NormalTok{) }\OperatorTok{+}
\StringTok{  }\KeywordTok{theme_void}\NormalTok{() }\OperatorTok{+}
\StringTok{  }\CommentTok{# theme(legend.position="none") +}
\StringTok{  }\KeywordTok{scale_fill_distiller}\NormalTok{(}\DataTypeTok{palette =} \StringTok{"Blues"}\NormalTok{)}

\CommentTok{# tree map}
\KeywordTok{ggraph}\NormalTok{(mygraph, }\StringTok{'treemap'}\NormalTok{, }\DataTypeTok{weight =} \StringTok{'size'}\NormalTok{) }\OperatorTok{+}\StringTok{ }
\StringTok{  }\KeywordTok{geom_node_tile}\NormalTok{(}\KeywordTok{aes}\NormalTok{(}\DataTypeTok{fill =}\NormalTok{ depth), }\DataTypeTok{size =} \FloatTok{0.25}\NormalTok{) }\OperatorTok{+}
\StringTok{  }\KeywordTok{theme_void}\NormalTok{() }\OperatorTok{+}
\StringTok{  }\KeywordTok{theme}\NormalTok{(}\DataTypeTok{legend.position=}\StringTok{"none"}\NormalTok{)}

\CommentTok{# circular partition}
\KeywordTok{ggraph}\NormalTok{(mygraph, }\StringTok{'partition'}\NormalTok{, }\DataTypeTok{circular =} \OtherTok{TRUE}\NormalTok{) }\OperatorTok{+}\StringTok{ }
\StringTok{  }\KeywordTok{geom_node_arc_bar}\NormalTok{(}\KeywordTok{aes}\NormalTok{(}\DataTypeTok{fill =}\NormalTok{ depth), }\DataTypeTok{size =} \FloatTok{0.25}\NormalTok{) }\OperatorTok{+}
\StringTok{  }\KeywordTok{theme_void}\NormalTok{() }\OperatorTok{+}
\StringTok{  }\KeywordTok{theme}\NormalTok{(}\DataTypeTok{legend.position=}\StringTok{"none"}\NormalTok{)}

\CommentTok{# node }
\KeywordTok{ggraph}\NormalTok{(mygraph) }\OperatorTok{+}\StringTok{ }
\StringTok{  }\KeywordTok{geom_edge_link}\NormalTok{() }\OperatorTok{+}\StringTok{ }
\StringTok{  }\KeywordTok{geom_node_point}\NormalTok{() }\OperatorTok{+}
\StringTok{  }\KeywordTok{theme_void}\NormalTok{() }\OperatorTok{+}
\StringTok{  }\KeywordTok{theme}\NormalTok{(}\DataTypeTok{legend.position=}\StringTok{"none"}\NormalTok{)}
\end{Highlighting}
\end{Shaded}

Insert an animal silhouette into a plot

\begin{Shaded}
\begin{Highlighting}[]
\CommentTok{#1. Get image from http://www.phylopic.org  }
\KeywordTok{library}\NormalTok{(png)}
\NormalTok{ima <-}\StringTok{ }\KeywordTok{readPNG}\NormalTok{(}\StringTok{"thething.png"}\NormalTok{)}
\KeywordTok{plot}\NormalTok{(}\DecValTok{1}\OperatorTok{:}\DecValTok{3}\NormalTok{,}\DecValTok{1}\OperatorTok{:}\DecValTok{3}\NormalTok{)}
\KeywordTok{rasterImage}\NormalTok{(}\DataTypeTok{image=}\NormalTok{ima, }\DataTypeTok{xleft=}\DecValTok{2}\NormalTok{,}\DataTypeTok{ybottom=}\FloatTok{1.8}\NormalTok{,}
            \DataTypeTok{xright=}\FloatTok{2.7}\NormalTok{,}\DataTypeTok{ytop=}\FloatTok{2.7}\NormalTok{)}
\end{Highlighting}
\end{Shaded}

Create an empty plot window

\begin{Shaded}
\begin{Highlighting}[]
\CommentTok{# 1}
\KeywordTok{plot}\NormalTok{(}\DecValTok{0}\NormalTok{,}\DataTypeTok{type=}\StringTok{'n'}\NormalTok{,}\DataTypeTok{axes=}\OtherTok{FALSE}\NormalTok{,}\DataTypeTok{ann=}\OtherTok{FALSE}\NormalTok{)  }
\CommentTok{# 2  }
\KeywordTok{plot}\NormalTok{(}\DecValTok{1}\NormalTok{, }\DataTypeTok{type=}\StringTok{"n"}\NormalTok{, }\DataTypeTok{xlab=}\StringTok{""}\NormalTok{, }\DataTypeTok{ylab=}\StringTok{""}\NormalTok{, }\DataTypeTok{xlim=}\KeywordTok{c}\NormalTok{(}\DecValTok{0}\NormalTok{, }\DecValTok{10}\NormalTok{), }\DataTypeTok{ylim=}\KeywordTok{c}\NormalTok{(}\DecValTok{0}\NormalTok{, }\DecValTok{10}\NormalTok{))  }
\CommentTok{# 3  }
\KeywordTok{plot.new}\NormalTok{()  }
\end{Highlighting}
\end{Shaded}

Set color gradient, palette for smoothing data points

\begin{Shaded}
\begin{Highlighting}[]
\KeywordTok{require}\NormalTok{(RColorBrewer)}

\NormalTok{alpha <-}\StringTok{ }\FloatTok{0.8} \CommentTok{# transparency (0 to 1 value)}
\KeywordTok{set.seed}\NormalTok{(}\DecValTok{5000}\NormalTok{)}
\NormalTok{rr <-}\StringTok{ }\KeywordTok{rnorm}\NormalTok{(}\DecValTok{5000}\NormalTok{)}

\CommentTok{# user defined gradient}
\NormalTok{col<-}\KeywordTok{colorRampPalette}\NormalTok{(}\KeywordTok{c}\NormalTok{(}\StringTok{"steelblue"}\NormalTok{,}\StringTok{"lightblue"}\NormalTok{,}\StringTok{"orange"}\NormalTok{,}\StringTok{"red"}\NormalTok{)) }\CommentTok{# set your own col gradient with as many colours as you want}
\NormalTok{colfunc <-}\StringTok{ }\KeywordTok{col}\NormalTok{(}\KeywordTok{length}\NormalTok{(rr))[}\KeywordTok{as.numeric}\NormalTok{(}\KeywordTok{cut}\NormalTok{(rr,}\DataTypeTok{breaks =} \KeywordTok{length}\NormalTok{(rr)))] }\CommentTok{# define breaks in col gradient}
\KeywordTok{plot}\NormalTok{(rr,}\DataTypeTok{col=}\NormalTok{colfunc,}\DataTypeTok{pch=}\DecValTok{20}\NormalTok{)}

\CommentTok{# gradient from palette}
\KeywordTok{display.brewer.all}\NormalTok{()}
\NormalTok{col <-}\StringTok{ "Greens"}
\NormalTok{col<-}\KeywordTok{colorRampPalette}\NormalTok{(}\KeywordTok{brewer.pal}\NormalTok{(brewer.pal.info[col,]}\OperatorTok{$}\NormalTok{maxcolors,col)) }\CommentTok{# col gradient}
\NormalTok{colfunc <-}\StringTok{ }\KeywordTok{col}\NormalTok{(}\KeywordTok{length}\NormalTok{(rr))[}\KeywordTok{as.numeric}\NormalTok{(}\KeywordTok{cut}\NormalTok{(rr,}\DataTypeTok{breaks =} \KeywordTok{length}\NormalTok{(rr)))] }\CommentTok{# define breaks in col gradient}
\KeywordTok{plot}\NormalTok{(rr,}\DataTypeTok{col=}\NormalTok{colfunc,}\DataTypeTok{pch=}\DecValTok{20}\NormalTok{)}
\end{Highlighting}
\end{Shaded}

Add plot point every nth element

\begin{Shaded}
\begin{Highlighting}[]
\NormalTok{n <-}\StringTok{ }\DecValTok{3}
\KeywordTok{plot}\NormalTok{(}\KeywordTok{runif}\NormalTok{(}\DecValTok{10}\NormalTok{, }\DecValTok{0}\NormalTok{, }\DecValTok{1}\NormalTok{), }\DataTypeTok{type =} \StringTok{"o"}\NormalTok{, }\DataTypeTok{pch =} \KeywordTok{c}\NormalTok{(}\DecValTok{20}\NormalTok{, }\KeywordTok{rep}\NormalTok{(}\OtherTok{NA}\NormalTok{, n)))}
\end{Highlighting}
\end{Shaded}

Create function to make line as default type in plot

\begin{Shaded}
\begin{Highlighting}[]
\NormalTok{lplot <-}\StringTok{ }\ControlFlowTok{function}\NormalTok{(...) }\KeywordTok{plot}\NormalTok{(..., }\DataTypeTok{type=}\StringTok{"l"}\NormalTok{)}
\KeywordTok{lplot}\NormalTok{(}\KeywordTok{runif}\NormalTok{(}\DecValTok{200}\NormalTok{))}
\end{Highlighting}
\end{Shaded}

Stack dataframe columns automatically in plot

\begin{Shaded}
\begin{Highlighting}[]
\KeywordTok{head}\NormalTok{(outplot)}
  \CommentTok{# time          N        P        S        I}
\CommentTok{# 1 0.00 200.000000 200.0000 20.00000 2.000000}
\CommentTok{# 2 0.01  78.245140 177.1952 20.58217 2.067159}
\CommentTok{# 3 0.02  34.785145 168.9650 21.12174 2.136073}
\NormalTok{dats <-}\StringTok{ }\KeywordTok{zoo}\NormalTok{(outplot)}
\KeywordTok{plot}\NormalTok{(dats)}
\end{Highlighting}
\end{Shaded}

Make 3D scatterplot

\begin{Shaded}
\begin{Highlighting}[]
\KeywordTok{require}\NormalTok{(scatterplot3d)}
\NormalTok{xx <-}\StringTok{ }\KeywordTok{rnorm}\NormalTok{(}\DecValTok{1000}\NormalTok{)}
\NormalTok{yy <-}\StringTok{ }\KeywordTok{runif}\NormalTok{(}\DecValTok{1000}\NormalTok{)}
\NormalTok{dens <-}\StringTok{ }\KeywordTok{c}\NormalTok{(}\KeywordTok{rep}\NormalTok{(}\FloatTok{0.0001}\NormalTok{,}\DecValTok{500}\NormalTok{),}\KeywordTok{rep}\NormalTok{(}\DecValTok{1}\NormalTok{,}\DecValTok{500}\NormalTok{))}
\NormalTok{controls <-}\StringTok{ }\KeywordTok{runif}\NormalTok{(}\DecValTok{3}\NormalTok{)}
\NormalTok{add.control <-}\StringTok{ }\DecValTok{1}
\NormalTok{dens_val <-}\StringTok{ }\DecValTok{1}\OperatorTok{*}\DecValTok{10}\OperatorTok{^-}\DecValTok{10} \CommentTok{# 0 or 1*10^-10. value to knock out blanket of colour on plot surface }
\CommentTok{#linear model of r/ship between coords}
\NormalTok{dens_lm <-}\StringTok{ }\KeywordTok{lm}\NormalTok{(dens }\OperatorTok{~}\StringTok{ }\NormalTok{xx }\OperatorTok{+}\StringTok{ }\NormalTok{yy)}

\NormalTok{xlim <-}\StringTok{ }\KeywordTok{c}\NormalTok{(}\KeywordTok{min}\NormalTok{(xx),}\KeywordTok{max}\NormalTok{(xx)); ylim <-}\StringTok{ }\KeywordTok{c}\NormalTok{(}\KeywordTok{min}\NormalTok{(yy),}\KeywordTok{max}\NormalTok{(yy)); zlim=}\KeywordTok{c}\NormalTok{(}\KeywordTok{min}\NormalTok{(dens),}\KeywordTok{max}\NormalTok{(dens)) }\CommentTok{# set lims}
\NormalTok{colv <-}\StringTok{ "Blues"}
\NormalTok{colvv<-}\KeywordTok{colorRampPalette}\NormalTok{(}\KeywordTok{brewer.pal}\NormalTok{(brewer.pal.info[colv,]}\OperatorTok{$}\NormalTok{maxcolors,colv)) }\CommentTok{# col gradient}
\NormalTok{colvv<-}\KeywordTok{colorRampPalette}\NormalTok{(}\KeywordTok{c}\NormalTok{(}\StringTok{"steelblue"}\NormalTok{,}\StringTok{"lightblue"}\NormalTok{,}\StringTok{"orange"}\NormalTok{,}\StringTok{"red"}\NormalTok{)) }\CommentTok{# set your own col gradient with as many colours as you want}
\CommentTok{# colvv<-colorRampPalette(magma(length(dens))) # set your own col gradient with as many colours as you want}

\CommentTok{# set col palette}
\NormalTok{colfunc <-}\StringTok{ }\KeywordTok{colvv}\NormalTok{(}\KeywordTok{length}\NormalTok{(dens))[}\KeywordTok{as.numeric}\NormalTok{(}\KeywordTok{cut}\NormalTok{(dens,}\DataTypeTok{breaks =} \KeywordTok{length}\NormalTok{(dens)))] }\CommentTok{# define breaks in col gradient}
\NormalTok{bg <-}\StringTok{ }\KeywordTok{bpy.colors}\NormalTok{(}\DecValTok{1}\NormalTok{)}
\NormalTok{alpha <-}\StringTok{ }\FloatTok{0.8}

\CommentTok{# pdf(paste0(plot.dir,strat,"_",density,"_",stage,"_kudspdf.pdf"),width=8.27,height=11.69,paper="a4r")}
\KeywordTok{scatterplot3d}\NormalTok{(}\DataTypeTok{x=}\NormalTok{xx,}\DataTypeTok{y=}\NormalTok{yy,}\DataTypeTok{z=}\NormalTok{dens,}
              \CommentTok{# color=ifelse(col_heat==1, adjustcolor(colfunc, alpha=1),adjustcolor("lightgreen",alpha=0.2)),}
              \DataTypeTok{color=}\KeywordTok{ifelse}\NormalTok{(dens}\OperatorTok{<=}\NormalTok{dens_val,}\KeywordTok{adjustcolor}\NormalTok{(}\KeywordTok{ifelse}\NormalTok{(bg}\OperatorTok{==}\KeywordTok{bpy.colors}\NormalTok{(}\DecValTok{1}\NormalTok{),}\KeywordTok{bpy.colors}\NormalTok{(}\DecValTok{1}\NormalTok{),}\StringTok{"white"}\NormalTok{),}\DataTypeTok{alpha=}\FloatTok{0.1}\NormalTok{),}\KeywordTok{adjustcolor}\NormalTok{(colfunc,}\DataTypeTok{alpha=}\NormalTok{alpha)),}
              \CommentTok{# col.axis="light green",}
              \DataTypeTok{las=}\DecValTok{1}\NormalTok{,}
              \DataTypeTok{pch=}\DecValTok{15}\NormalTok{,}
              \DataTypeTok{type=}\StringTok{"p"}\NormalTok{,}
              \DataTypeTok{lty.hplot =} \DecValTok{1}\NormalTok{,}
              \DataTypeTok{xlim=}\NormalTok{xlim,}
              \DataTypeTok{ylim=}\NormalTok{ylim,}
              \DataTypeTok{zlim=}\NormalTok{zlim,}
              \DataTypeTok{xlab=}\StringTok{"X"}\NormalTok{,}
              \DataTypeTok{ylab=}\StringTok{"Y"}\NormalTok{,}
              \DataTypeTok{zlab=}\StringTok{"Density"}\NormalTok{,}
              \DataTypeTok{main=}\StringTok{"Main"}\NormalTok{,}
              \DataTypeTok{box=}\NormalTok{F,}
              \DataTypeTok{lty.axis=}\KeywordTok{par}\NormalTok{(}\DecValTok{1}\NormalTok{),}
              \DataTypeTok{grid=}\NormalTok{F,}
              \DataTypeTok{col.grid =} \KeywordTok{adjustcolor}\NormalTok{(}\StringTok{"gray"}\NormalTok{,}\DecValTok{1}\NormalTok{),}
              \DataTypeTok{lty.grid=}\KeywordTok{par}\NormalTok{(}\DecValTok{3}\NormalTok{),}
              \CommentTok{#cex.symbols=dens*3,}
              \CommentTok{#cex.symbols = ifelse(z<=0,0,0.5),}
              \CommentTok{# highlight.3d=T, # ignores color arg if T}
              \CommentTok{# angle=70,}
              \DataTypeTok{axis=}\NormalTok{T}
              \CommentTok{# add below part to end of scatterplot3d plot}
\NormalTok{)}\CommentTok{#$plane3d(dens_lm, # add 3d linear model plane. # ??plane3d(Intercept, x.coef = NULL, y.coef = NULL, lty = "dashed", lty.box = NULL, draw_lines = TRUE, draw_polygon = FALSE, polygon_args = list(border = NA, col = rgb(0,0,0,0.2))}
\CommentTok{#            lty="dashed",}
\CommentTok{#          lty.box = NULL,}
\CommentTok{#          draw_lines = F, draw_polygon = T,}
\CommentTok{#          polygon_args = list(border = NA, col = adjustcolor("light green",alpha=0.4)))}
\CommentTok{# add control dates}
\ControlFlowTok{if}\NormalTok{(add.control}\OperatorTok{==}\DecValTok{1}\NormalTok{)\{}\KeywordTok{par}\NormalTok{(}\DataTypeTok{new=}\NormalTok{T); }\KeywordTok{scatterplot3d}\NormalTok{(}\DataTypeTok{x=}\KeywordTok{rep}\NormalTok{(}\DecValTok{0}\NormalTok{,}\KeywordTok{length}\NormalTok{(controls)),}\DataTypeTok{y=}\NormalTok{controls,}\DataTypeTok{z=}\KeywordTok{rep}\NormalTok{(}\KeywordTok{max}\NormalTok{(dens),}\KeywordTok{length}\NormalTok{(controls)),}\DataTypeTok{color=}\StringTok{"gray"}\NormalTok{,}\DataTypeTok{las=}\DecValTok{1}\NormalTok{,}\DataTypeTok{pch=}\StringTok{""}\NormalTok{,}\DataTypeTok{lty.hplot =} \DecValTok{1}\NormalTok{,}\DataTypeTok{xlim=}\NormalTok{xlim,}\DataTypeTok{ylim=}\NormalTok{ylim,}\DataTypeTok{zlim=}\NormalTok{zlim,}\DataTypeTok{xlab=}\StringTok{""}\NormalTok{,}\DataTypeTok{ylab=}\StringTok{""}\NormalTok{,}\DataTypeTok{zlab=}\StringTok{""}\NormalTok{,}\DataTypeTok{box=}\NormalTok{F,}\DataTypeTok{grid=}\NormalTok{F,}\DataTypeTok{cex.symbols=}\DecValTok{2}\NormalTok{,}\DataTypeTok{axis=}\NormalTok{F,}\DataTypeTok{type=}\StringTok{"h"}\NormalTok{)\}}
\end{Highlighting}
\end{Shaded}

Adding title from separate list to plot in loop (\texttt{ggplot})

\begin{Shaded}
\begin{Highlighting}[]
\CommentTok{# plot all sim results in one window }
\NormalTok{gspl <-}\StringTok{ }\KeywordTok{list}\NormalTok{()}
\NormalTok{ttl_list <-}\StringTok{ }\KeywordTok{c}\NormalTok{(}\StringTok{"cerc"}\NormalTok{,}\StringTok{"food"}\NormalTok{, }\StringTok{"juv"}\NormalTok{, }\StringTok{"adult"}\NormalTok{, }\StringTok{"infec"}\NormalTok{, }\StringTok{"infec (shed)"}\NormalTok{, }\StringTok{"host L"}\NormalTok{, }\StringTok{"parasite mass"}\NormalTok{)}

\CommentTok{# choose sim to plot}
\NormalTok{global_sim_plot <-}\StringTok{ }\NormalTok{global_detritus}

\ControlFlowTok{for}\NormalTok{(g }\ControlFlowTok{in} \DecValTok{1}\OperatorTok{:}\DecValTok{10}\NormalTok{)\{}
\NormalTok{  gspl[[g]] <-}\StringTok{ }\KeywordTok{ggplot}\NormalTok{() }\OperatorTok{+}
\StringTok{  }\KeywordTok{geom_line}\NormalTok{(}\DataTypeTok{data =}\NormalTok{ y_m, }\KeywordTok{aes}\NormalTok{(}\DataTypeTok{x =} \KeywordTok{rep.int}\NormalTok{(}\DecValTok{1}\OperatorTok{:}\NormalTok{n.ticks,}\KeywordTok{max}\NormalTok{(L1)) , }\DataTypeTok{y =}\NormalTok{ value, }\DataTypeTok{group =}\NormalTok{ L1, }\DataTypeTok{colour=}\KeywordTok{factor}\NormalTok{(L1)), ) }\OperatorTok{+}
\StringTok{  }\CommentTok{# scale_color_manual(values = viridis(length(mm))) +}
\StringTok{  }\CommentTok{#linetype=y_m$L1) +}
\StringTok{  }\KeywordTok{theme_tufte}\NormalTok{() }\OperatorTok{+}
\StringTok{    }\KeywordTok{labs}\NormalTok{(}\DataTypeTok{title=}\NormalTok{ttl_list[g],}\DataTypeTok{x=}\StringTok{""}\NormalTok{,}\DataTypeTok{y=}\StringTok{""}\NormalTok{) }\OperatorTok{+}
\StringTok{    }\ControlFlowTok{if}\NormalTok{(g}\OperatorTok{==}\KeywordTok{length}\NormalTok{(global_sim_plot))\{}
      \KeywordTok{theme}\NormalTok{(}\DataTypeTok{legend.title=}\KeywordTok{element_text}\NormalTok{(}\DataTypeTok{size=}\FloatTok{0.2}\NormalTok{), }
            \DataTypeTok{legend.text=}\KeywordTok{element_text}\NormalTok{(}\DataTypeTok{size=}\FloatTok{0.2}\NormalTok{)) }\OperatorTok{+}
\StringTok{        }\KeywordTok{theme}\NormalTok{(}\DataTypeTok{legend.position =} \StringTok{"top"}\NormalTok{)}
      \KeywordTok{labs}\NormalTok{(}\DataTypeTok{x=}\StringTok{"Time"}\NormalTok{)}
\NormalTok{    \}}\ControlFlowTok{else}\NormalTok{\{}
      \KeywordTok{theme}\NormalTok{(}\DataTypeTok{legend.position=}\StringTok{"none"}\NormalTok{)}
\NormalTok{    \}}
\NormalTok{    \}}
\CommentTok{# +  geom_text(x=,y=,label = max(value),check_overlap = TUE)}
\KeywordTok{do.call}\NormalTok{(grid.arrange,gspl) }\CommentTok{# plot in one window }
\end{Highlighting}
\end{Shaded}

Using math expressions in plot labels

\begin{Shaded}
\begin{Highlighting}[]
\KeywordTok{plot}\NormalTok{(}\KeywordTok{rnorm}\NormalTok{(}\DecValTok{1000}\NormalTok{),}
     \DataTypeTok{xlab=}\KeywordTok{expression}\NormalTok{(}\KeywordTok{paste}\NormalTok{(}\StringTok{"X values"}\OperatorTok{^}\DecValTok{2}\NormalTok{)),}
     \DataTypeTok{ylab=}\KeywordTok{expression}\NormalTok{(}\KeywordTok{paste}\NormalTok{(}\StringTok{"Y values"}\OperatorTok{^}\DecValTok{3}\NormalTok{,}\KeywordTok{hat}\NormalTok{(beta)))}
\NormalTok{)}
\end{Highlighting}
\end{Shaded}

Adding faint gridlines to plot

\begin{Shaded}
\begin{Highlighting}[]
\CommentTok{# add gridlines}
\KeywordTok{grid}\NormalTok{(}\DataTypeTok{nx=}\OtherTok{NA}\NormalTok{,}\DataTypeTok{ny=}\OtherTok{NULL}\NormalTok{)}
\end{Highlighting}
\end{Shaded}

Storing current \texttt{par} variables for plotting

\begin{Shaded}
\begin{Highlighting}[]
\NormalTok{og_pars <-}\StringTok{ }\KeywordTok{par}\NormalTok{(}\DataTypeTok{no.readonly =}\NormalTok{ T) }\CommentTok{# store current par values}
\end{Highlighting}
\end{Shaded}

\subsubsection{Reading in files/data}\label{reading-in-filesdata}

Read in file manually

\begin{Shaded}
\begin{Highlighting}[]
\NormalTok{get.file.vol <-}\StringTok{ }\KeywordTok{read.table}\NormalTok{(}\KeywordTok{file.choose}\NormalTok{())}\CommentTok{#read file manually}
\NormalTok{v.file <-}\StringTok{ }\NormalTok{get.file.vol[}\DecValTok{1}\OperatorTok{:}\DecValTok{100}\NormalTok{,}\DecValTok{1}\NormalTok{]}\CommentTok{#get the volume}
\end{Highlighting}
\end{Shaded}

Loop through files from dir and append to list

\begin{Shaded}
\begin{Highlighting}[]
\CommentTok{# reading in spdf (hrpath) files from drive}
\KeywordTok{setwd}\NormalTok{(}\StringTok{"/Users/camel/Desktop/Matt2016/Manuscripts/MalishevBullKearney/Resubmission/2016/barcoo sims/barcooresults/hrpath_75"}\NormalTok{)}
\NormalTok{file.list<-}\KeywordTok{list.files}\NormalTok{()}
\NormalTok{hrs75<-}\KeywordTok{as.list}\NormalTok{(}\KeywordTok{rep}\NormalTok{(}\DecValTok{1}\NormalTok{,}\DecValTok{100}\NormalTok{)) }\CommentTok{# empty list}
\ControlFlowTok{for}\NormalTok{ (f }\ControlFlowTok{in} \DecValTok{1}\OperatorTok{:}\DecValTok{100}\NormalTok{)\{}
  \KeywordTok{load}\NormalTok{(file.list[f])}
\NormalTok{  hrs75[f]<-hrpath}
\NormalTok{\}}

\CommentTok{# working version}
\CommentTok{#converting spdf into mcp(spdf,100,unout="m2)}
\NormalTok{ghr<-}\KeywordTok{list}\NormalTok{()}
\ControlFlowTok{for}\NormalTok{ (i }\ControlFlowTok{in}\NormalTok{ hrs75[}\DecValTok{1}\OperatorTok{:}\DecValTok{10}\NormalTok{]) \{}
\NormalTok{  m<-}\KeywordTok{mcp}\NormalTok{(i,}\DecValTok{100}\NormalTok{,}\DataTypeTok{unout=}\StringTok{'m2'}\NormalTok{)}
\NormalTok{  ghr<-}\KeywordTok{c}\NormalTok{(ghr,m)}
\NormalTok{\};ghr}
\end{Highlighting}
\end{Shaded}

Read in PDF files from online source in R and save to drive.

\begin{Shaded}
\begin{Highlighting}[]
\CommentTok{# from https://github.com/ropensci/pdftools}

\KeywordTok{require}\NormalTok{(pdftools)}
\NormalTok{url <-}\StringTok{ "https://raw.githubusercontent.com/darwinanddavis/499R/master/exp_pop_growth.pdf"}
\NormalTok{dir <-}\StringTok{ "FOLDER ON YOUR COMPUTER WHERE YOU WANT THE FILE SAVED"}
\NormalTok{f <-}\StringTok{ "NAME OF THE FILE"} 
\NormalTok{f <-}\StringTok{ }\KeywordTok{paste0}\NormalTok{(f,}\StringTok{".pdf"}\NormalTok{) }

\CommentTok{# run all this}
\KeywordTok{download.file}\NormalTok{(url, }\KeywordTok{paste0}\NormalTok{(dir,}\StringTok{"/"}\NormalTok{,f), }\DataTypeTok{mode =} \StringTok{"wb"}\NormalTok{)}
\NormalTok{txt <-}\StringTok{ }\KeywordTok{pdf_text}\NormalTok{(}\KeywordTok{paste0}\NormalTok{(dir,}\StringTok{"/"}\NormalTok{,f))}

\CommentTok{# first page text}
\NormalTok{page <-}\StringTok{ }\DecValTok{1} \CommentTok{# enter the page number}
\KeywordTok{cat}\NormalTok{(txt[page])}

\NormalTok{toc <-}\StringTok{ }\KeywordTok{pdf_toc}\NormalTok{(}\KeywordTok{paste0}\NormalTok{(dir,}\StringTok{"/"}\NormalTok{,f))}

\KeywordTok{require}\NormalTok{(jsonlite)}
\CommentTok{# Show as JSON}
\NormalTok{jsonlite}\OperatorTok{::}\KeywordTok{toJSON}\NormalTok{(toc, }\DataTypeTok{auto_unbox =} \OtherTok{TRUE}\NormalTok{, }\DataTypeTok{pretty =} \OtherTok{TRUE}\NormalTok{)}

\CommentTok{# show author, version, etc}
\NormalTok{info <-}\StringTok{ }\KeywordTok{pdf_info}\NormalTok{(f)}

\CommentTok{# renders pdf to bitmap array}
\NormalTok{bitmap <-}\StringTok{ }\KeywordTok{pdf_render_page}\NormalTok{(f, }\DataTypeTok{page =} \DecValTok{1}\NormalTok{)}

\CommentTok{# save bitmap image}
\NormalTok{png}\OperatorTok{::}\KeywordTok{writePNG}\NormalTok{(bitmap, }\StringTok{"page.png"}\NormalTok{)}
\NormalTok{jpeg}\OperatorTok{::}\KeywordTok{writeJPEG}\NormalTok{(bitmap, }\StringTok{"page.jpeg"}\NormalTok{)}
\NormalTok{webp}\OperatorTok{::}\KeywordTok{write_webp}\NormalTok{(bitmap, }\StringTok{"page.webp"}\NormalTok{)}
\end{Highlighting}
\end{Shaded}

\subsubsection{\texorpdfstring{Regular expressions
(\texttt{regex})}{Regular expressions (regex)}}\label{regular-expressions-regex}

Get just numbers or characters

\begin{Shaded}
\begin{Highlighting}[]
\NormalTok{vec <-}\StringTok{ "16-Feb-2018 20:08:04 PM"}
\NormalTok{vecN <-}\StringTok{ }\KeywordTok{gsub}\NormalTok{(}\StringTok{"[^[:digit:]]"}\NormalTok{, }\StringTok{""}\NormalTok{, vec); vec; }\KeywordTok{print}\NormalTok{(}\KeywordTok{paste0}\NormalTok{(}\StringTok{"Just numbers: "}\NormalTok{,vecN)) }
\NormalTok{vecC <-}\StringTok{ }\KeywordTok{gsub}\NormalTok{(}\StringTok{"[[:digit:]]"}\NormalTok{, }\StringTok{""}\NormalTok{, vec); vec; }\KeywordTok{print}\NormalTok{(}\KeywordTok{paste0}\NormalTok{(}\StringTok{"Just characters: "}\NormalTok{, vecC))}

\CommentTok{# with tidyr. requires data frame }
\KeywordTok{require}\NormalTok{(tidyr)}
\NormalTok{df <-}\StringTok{ }\KeywordTok{data.frame}\NormalTok{(}\DataTypeTok{N1=}\KeywordTok{c}\NormalTok{(}\StringTok{"APPLE348744"}\NormalTok{, }\StringTok{"BANANA77845"}\NormalTok{, }\StringTok{"OATS2647892"}\NormalTok{, }\StringTok{"EGG98586456"}\NormalTok{))}
\KeywordTok{print}\NormalTok{(}\StringTok{"tidyr doesn't work with strings separated by spaces"}\NormalTok{)  }
\NormalTok{df }\OperatorTok
\StringTok{  }\KeywordTok{separate}\NormalTok{(N1, }\DataTypeTok{into =} \KeywordTok{c}\NormalTok{(}\StringTok{"text"}\NormalTok{, }\StringTok{"num"}\NormalTok{), }\DataTypeTok{sep =} \StringTok{"(?<=[A-Za-z])(?=[0-9])"}\NormalTok{)}
\end{Highlighting}
\end{Shaded}

Insert or replace a character in a string at a specific location

\begin{Shaded}
\begin{Highlighting}[]
\KeywordTok{require}\NormalTok{(stringi)}
\NormalTok{vec <-}\StringTok{ "ABCEF"}
\KeywordTok{stri_sub}\NormalTok{(vec, }\DecValTok{4}\NormalTok{, }\DecValTok{2}\NormalTok{) <-}\StringTok{ "d"} 
\KeywordTok{print}\NormalTok{(}\KeywordTok{paste0}\NormalTok{(}\StringTok{"Original: ABCEF"}\NormalTok{)); }\KeywordTok{print}\NormalTok{(}\KeywordTok{paste0}\NormalTok{(}\StringTok{"New: "}\NormalTok{,vec))}
\end{Highlighting}
\end{Shaded}

\begin{Shaded}
\begin{Highlighting}[]
\CommentTok{# Testing regex expressions and their output   }

\CommentTok{# https://regex101.com/r/ksY7HU/2}
\end{Highlighting}
\end{Shaded}

\subsubsection{R Markdown}\label{r-markdown}

Hide unwanted code output, such as inherent examples for functions

\begin{Shaded}
\begin{Highlighting}[]
\CommentTok{# ```\{r, cache = TRUE, tidy = TRUE, lazy = TRUE, results='markup'\}}
\end{Highlighting}
\end{Shaded}

Math notation in R Markdown

\begin{Shaded}
\begin{Highlighting}[]
\NormalTok{x=y }\OperatorTok{$}\NormalTok{x =}\StringTok{ }\NormalTok{y}\OperatorTok{$}\StringTok{    }
\NormalTok{x}\OperatorTok{<}\NormalTok{y }\OperatorTok{$}\NormalTok{x }\OperatorTok{<}\StringTok{ }\NormalTok{y}\OperatorTok{$}\StringTok{  }
\NormalTok{x}\OperatorTok{>}\NormalTok{y }\OperatorTok{$}\NormalTok{x }\OperatorTok{>}\StringTok{ }\NormalTok{y}\OperatorTok{$}\StringTok{  }
\NormalTok{x≤y }\OperatorTok{$}\NormalTok{x \textbackslash{}le y}\OperatorTok{$}\StringTok{  }
\NormalTok{x≥y }\OperatorTok{$}\NormalTok{x \textbackslash{}ge y}\OperatorTok{$}\StringTok{  }
\NormalTok{xn  }\OperatorTok{$}\NormalTok{x}\OperatorTok{^}\NormalTok{\{n\}}\OperatorTok{$}\StringTok{  }
\NormalTok{xn  }\OperatorTok{$}\NormalTok{x_\{n\}}\OperatorTok{$}\StringTok{  }
\NormalTok{x⎯⎯⎯    }\OperatorTok{$}\NormalTok{\textbackslash{}overline\{x\}}\OperatorTok{$}\StringTok{  }
\NormalTok{x̂  }\OperatorTok{$}\NormalTok{\textbackslash{}hat\{x\}}\OperatorTok{$}\StringTok{  }
\NormalTok{x̃  }\OperatorTok{$}\NormalTok{\textbackslash{}tilde\{x\}}\OperatorTok{$}\StringTok{  }
\NormalTok{ab  }\OperatorTok{$}\NormalTok{\textbackslash{}frac\{a\}\{b\}}\OperatorTok{$}\StringTok{  }
\NormalTok{∂f∂x    }\OperatorTok{$}\NormalTok{\textbackslash{}frac\{a\}\{b\}}\OperatorTok{$}\StringTok{  }
\NormalTok{∂f∂x    }\OperatorTok{$}\NormalTok{\textbackslash{}displaystyle \textbackslash{}frac\{a\}\{b\}}\OperatorTok{$}\StringTok{  }
\NormalTok{(nk)    }\OperatorTok{$}\NormalTok{\textbackslash{}binom\{n\}\{k\}}\OperatorTok{$}\StringTok{  }
\NormalTok{x1}\OperatorTok{+}\NormalTok{x2}\OperatorTok{+}\NormalTok{⋯}\OperatorTok{+}\NormalTok{xn  }\OperatorTok{$}\NormalTok{x_\{}\DecValTok{1}\NormalTok{\} }\OperatorTok{+}\StringTok{ }\NormalTok{x_\{}\DecValTok{2}\NormalTok{\} }\OperatorTok{+}\StringTok{ }\NormalTok{\textbackslash{}cdots }\OperatorTok{+}\StringTok{ }\NormalTok{x_\{n\}}\OperatorTok{$}\StringTok{   }
\NormalTok{x1,x2,…,xn  }\OperatorTok{$}\NormalTok{x_\{}\DecValTok{1}\NormalTok{\}, x_\{}\DecValTok{2}\NormalTok{\}, \textbackslash{}dots, x_\{n\}}\OperatorTok{$}\StringTok{  }
\NormalTok{x=⟨x1,x2,…,xn }\OperatorTok{$}\NormalTok{\textbackslash{}mathbf\{x\} =}\StringTok{ }\NormalTok{\textbackslash{}langle x_\{}\DecValTok{1}\NormalTok{\}, x_\{}\DecValTok{2}\NormalTok{\}, \textbackslash{}dots, x_\{n\}\textbackslash{}rangle}\OperatorTok{$}\StringTok{    }
\NormalTok{x∈A }\OperatorTok{$}\NormalTok{x \textbackslash{}}\ControlFlowTok{in}\NormalTok{ A}\OperatorTok{$}\StringTok{    }
\ErrorTok{|}\NormalTok{A}\OperatorTok{|}\StringTok{ }\ErrorTok{$|}\NormalTok{A}\OperatorTok{|}\ErrorTok{$}\StringTok{  }
\NormalTok{x∈A }\OperatorTok{$}\NormalTok{x \textbackslash{}}\ControlFlowTok{in}\NormalTok{ A}\OperatorTok{$}\StringTok{  }
\NormalTok{A⊂B }\OperatorTok{$}\NormalTok{x \textbackslash{}subset B}\OperatorTok{$}\StringTok{  }
\NormalTok{A⊆B }\OperatorTok{$}\NormalTok{x \textbackslash{}subseteq B}\OperatorTok{$}\StringTok{  }
\NormalTok{A∪B }\OperatorTok{$}\NormalTok{A \textbackslash{}cup B}\OperatorTok{$}\StringTok{  }
\NormalTok{A∩B }\OperatorTok{$}\NormalTok{A \textbackslash{}cap B}\OperatorTok{$}\StringTok{  }
\NormalTok{X∼𝖡𝗂𝗇𝗈𝗆(n,π)    }\OperatorTok{$}\NormalTok{X \textbackslash{}sim \{\textbackslash{}sf Binom\}(n, \textbackslash{}pi)}\OperatorTok{$}

\KeywordTok{P}\NormalTok{(X≤x)=𝚙𝚋𝚒𝚗𝚘𝚖(x,n,π)    }\OperatorTok{$}\NormalTok{\textbackslash{}mathrm\{P\}(X \textbackslash{}le x) =}\StringTok{ }\NormalTok{\{\textbackslash{}tt pbinom\}(x, n, \textbackslash{}pi)}\OperatorTok{$}\StringTok{   }
\KeywordTok{P}\NormalTok{(A∣B)  }\OperatorTok{$}\KeywordTok{P}\NormalTok{(A \textbackslash{}mid B)}\OperatorTok{$}\StringTok{  }
\KeywordTok{P}\NormalTok{(A∣B)  }\OperatorTok{$}\NormalTok{\textbackslash{}mathrm\{P\}(A \textbackslash{}mid B)}\OperatorTok{$}\StringTok{   }
\NormalTok{\{}\DecValTok{1}\NormalTok{,}\DecValTok{2}\NormalTok{,}\DecValTok{3}\NormalTok{\} }\OperatorTok{$}\NormalTok{\textbackslash{}\{}\DecValTok{1}\NormalTok{, }\DecValTok{2}\NormalTok{, }\DecValTok{3}\NormalTok{\textbackslash{}\}}\OperatorTok{$}\StringTok{  }
\KeywordTok{sin}\NormalTok{(x)  }\OperatorTok{$}\NormalTok{\textbackslash{}}\KeywordTok{sin}\NormalTok{(x)}\OperatorTok{$}\StringTok{    }
\KeywordTok{log}\NormalTok{(x)  }\OperatorTok{$}\NormalTok{\textbackslash{}}\KeywordTok{log}\NormalTok{(x)}\OperatorTok{$}\StringTok{  }
\NormalTok{∫ba }\OperatorTok{$}\NormalTok{\textbackslash{}int_\{a\}}\OperatorTok{^}\NormalTok{\{b\}}\OperatorTok{$}\StringTok{  }
\NormalTok{(∫}\KeywordTok{baf}\NormalTok{(x)dx) }\OperatorTok{$}\NormalTok{\textbackslash{}}\KeywordTok{left}\NormalTok{(\textbackslash{}int_\{a\}}\OperatorTok{^}\NormalTok{\{b\} }\KeywordTok{f}\NormalTok{(x) \textbackslash{}; dx\textbackslash{}right)}\OperatorTok{$}\StringTok{  }
\NormalTok{[∫∞−∞}\KeywordTok{f}\NormalTok{(x)dx]    }\OperatorTok{$}\NormalTok{\textbackslash{}left[\textbackslash{}int_\{\textbackslash{}}\OperatorTok{-}\NormalTok{infty\}}\OperatorTok{^}\NormalTok{\{\textbackslash{}infty\} }\KeywordTok{f}\NormalTok{(x) \textbackslash{}; dx\textbackslash{}right]}\OperatorTok{$}\StringTok{  }
\KeywordTok{F}\NormalTok{(x)}\OperatorTok{|}\NormalTok{ba }\OperatorTok{$}\NormalTok{\textbackslash{}left. }\KeywordTok{F}\NormalTok{(x) \textbackslash{}right}\OperatorTok{|}\NormalTok{_\{a\}}\OperatorTok{^}\NormalTok{\{b\}}\OperatorTok{$}\StringTok{  }
\NormalTok{∑bx=}\KeywordTok{af}\NormalTok{(x)   }\OperatorTok{$}\NormalTok{\textbackslash{}sum_\{x =}\StringTok{ }\NormalTok{a\}}\OperatorTok{^}\NormalTok{\{b\} }\KeywordTok{f}\NormalTok{(x)}\OperatorTok{$}\StringTok{  }
\NormalTok{∏bx=}\KeywordTok{af}\NormalTok{(x)   }\OperatorTok{$}\NormalTok{\textbackslash{}prod_\{x =}\StringTok{ }\NormalTok{a\}}\OperatorTok{^}\NormalTok{\{b\} }\KeywordTok{f}\NormalTok{(x)}\OperatorTok{$}\StringTok{  }
\NormalTok{limx→∞}\KeywordTok{f}\NormalTok{(x)  }\OperatorTok{$}\NormalTok{\textbackslash{}lim_\{x \textbackslash{}to \textbackslash{}infty\} }\KeywordTok{f}\NormalTok{(x)}\OperatorTok{$}\StringTok{  }
\NormalTok{limx→∞}\KeywordTok{f}\NormalTok{(x)  }\OperatorTok{$}\NormalTok{\textbackslash{}displaystyle \textbackslash{}lim_\{x \textbackslash{}to \textbackslash{}infty\} }\KeywordTok{f}\NormalTok{(x)}\OperatorTok{$}\StringTok{    }
\end{Highlighting}
\end{Shaded}

Greek Letters

\begin{Shaded}
\begin{Highlighting}[]
\NormalTok{αA  }\OperatorTok{$}\NormalTok{\textbackslash{}alpha A}\OperatorTok{$}\StringTok{      }
\NormalTok{νN  }\OperatorTok{$}\NormalTok{\textbackslash{}nu N }\OperatorTok{$}\StringTok{  }
\NormalTok{βB  }\OperatorTok{$}\NormalTok{\textbackslash{}beta B}\OperatorTok{$}\StringTok{   }
\NormalTok{ξΞ  }\OperatorTok{$}\NormalTok{\textbackslash{}xi\textbackslash{}Xi}\OperatorTok{$}\StringTok{  }
\NormalTok{γΓ  }\OperatorTok{$}\NormalTok{\textbackslash{}gamma \textbackslash{}Gamma}\OperatorTok{$}\StringTok{  }
\NormalTok{oO  }\OperatorTok{$}\NormalTok{o O}\OperatorTok{$}\StringTok{ }\NormalTok{(omicron)  }
\NormalTok{δΔ  }\OperatorTok{$}\NormalTok{\textbackslash{}delta \textbackslash{}Delta}\OperatorTok{$}\StringTok{   }
\NormalTok{πΠ  }\OperatorTok{$}\NormalTok{\textbackslash{}pi \textbackslash{}Pi}\OperatorTok{$}\StringTok{  }
\NormalTok{ϵεE }\OperatorTok{$}\NormalTok{\textbackslash{}epsilon \textbackslash{}varepsilon E}\OperatorTok{$}\StringTok{  }
\NormalTok{ρϱP }\OperatorTok{$}\NormalTok{\textbackslash{}rho\textbackslash{}varrho P}\OperatorTok{$}\StringTok{  }
\NormalTok{ζZ  }\OperatorTok{$}\NormalTok{\textbackslash{}zeta Z \textbackslash{}sigma \textbackslash{},\textbackslash{}}\OperatorTok{!}\ErrorTok{$}\StringTok{  }
\NormalTok{Σ   }\OperatorTok{$}\NormalTok{\textbackslash{}sigma \textbackslash{}Sigma}\OperatorTok{$}\StringTok{  }
\NormalTok{ηH  }\OperatorTok{$}\NormalTok{\textbackslash{}eta H}\OperatorTok{$}\StringTok{  }
\NormalTok{τT  }\OperatorTok{$}\NormalTok{\textbackslash{}tau T}\OperatorTok{$}\StringTok{  }
\NormalTok{θϑΘ }\OperatorTok{$}\NormalTok{\textbackslash{}theta \textbackslash{}vartheta \textbackslash{}Theta}\OperatorTok{$}\StringTok{  }
\NormalTok{υΥ  }\OperatorTok{$}\NormalTok{\textbackslash{}upsilon \textbackslash{}Upsilon}\OperatorTok{$}\StringTok{  }
\NormalTok{ιI  }\OperatorTok{$}\NormalTok{\textbackslash{}iota I}\OperatorTok{$}\StringTok{  }
\NormalTok{ϕφΦ }\OperatorTok{$}\NormalTok{\textbackslash{}phi \textbackslash{}varphi \textbackslash{}Phi}\OperatorTok{$}\StringTok{    }
\NormalTok{κK  }\OperatorTok{$}\NormalTok{\textbackslash{}kappa K}\OperatorTok{$}\StringTok{    }
\NormalTok{χX  }\OperatorTok{$}\NormalTok{\textbackslash{}chi X}\OperatorTok{$}\StringTok{    }
\NormalTok{λΛ  }\OperatorTok{$}\NormalTok{\textbackslash{}lambda \textbackslash{}Lambda}\OperatorTok{$}\StringTok{    }
\NormalTok{ψΨ  }\OperatorTok{$}\NormalTok{\textbackslash{}psi \textbackslash{}Psi}\OperatorTok{$}\StringTok{    }
\NormalTok{μM  }\OperatorTok{$}\NormalTok{\textbackslash{}mu M}\OperatorTok{$}\StringTok{     }
\NormalTok{ω Ω}\OperatorTok{$}\NormalTok{\textbackslash{}omega \textbackslash{}Omega}\OperatorTok{$}\StringTok{    }
\end{Highlighting}
\end{Shaded}

\subsubsection{Subsetting}\label{subsetting}

Select specific rows E.g. select rows of sfeed\_move not in foodh

\begin{Shaded}
\begin{Highlighting}[]
\KeywordTok{library}\NormalTok{(sqldf)}
\NormalTok{a1NotIna2_h  <-}\StringTok{ }\KeywordTok{sqldf}\NormalTok{(}\StringTok{'SELECT * FROM sfeed_move EXCEPT SELECT * FROM foodh'}\NormalTok{)}
\NormalTok{a1NotIna2_l  <-}\StringTok{ }\KeywordTok{sqldf}\NormalTok{(}\StringTok{'SELECT * FROM sfeed_move EXCEPT SELECT * FROM foodl'}\NormalTok{)}
\CommentTok{# select rows from sfeed_move that also appear in foodh}
\NormalTok{a1Ina2_h  <-}\StringTok{ }\KeywordTok{sqldf}\NormalTok{(}\StringTok{'SELECT * FROM sfeed_move INTERSECT SELECT * FROM foodh'}\NormalTok{)}
\NormalTok{a1Ina2_l  <-}\StringTok{ }\KeywordTok{sqldf}\NormalTok{(}\StringTok{'SELECT * FROM sfeed_move INTERSECT SELECT * FROM foodl'}\NormalTok{)}
\end{Highlighting}
\end{Shaded}

Count occurrences of values in data frame

\begin{Shaded}
\begin{Highlighting}[]
\KeywordTok{table}\NormalTok{(}\KeywordTok{unlist}\NormalTok{(df}\OperatorTok{$}\NormalTok{X))}
\end{Highlighting}
\end{Shaded}

Remove a specific column from a data frame

\begin{Shaded}
\begin{Highlighting}[]
\KeywordTok{within}\NormalTok{(df, }\KeywordTok{rm}\NormalTok{(}\StringTok{"Col1"}\NormalTok{))}
\end{Highlighting}
\end{Shaded}

\subsubsection{Web scraping}\label{web-scraping}

Scraping web tables\\
\url{http://web.mit.edu/~r/current/arch/i386_linux26/lib/R/library/XML/html/readHTMLTable.html\%5Bhttp://web.mit.edu/~r/current/arch/i386_linux26/lib/R/library/XML/html/readHTMLTable.html\%5D}

\begin{Shaded}
\begin{Highlighting}[]
\KeywordTok{library}\NormalTok{(XML)}
\KeywordTok{readHTMLTable}\NormalTok{()}
\end{Highlighting}
\end{Shaded}

Scraping Twitter timelines\\
See complete example at
\url{http://varianceexplained.org/r/trump-tweets/}

\begin{Shaded}
\begin{Highlighting}[]
\CommentTok{# https://cran.r-project.org/web/packages/twitteR/}
\KeywordTok{library}\NormalTok{(dplyr)}
\KeywordTok{library}\NormalTok{(purrr)}
\KeywordTok{library}\NormalTok{(twitteR)}
\end{Highlighting}
\end{Shaded}

\printbibliography

\end{document}
